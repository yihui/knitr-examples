\documentclass[a4paper,titlepage]{tufte-handout}\usepackage[]{graphicx}\usepackage[]{color}
%% maxwidth is the original width if it is less than linewidth
%% otherwise use linewidth (to make sure the graphics do not exceed the margin)
\makeatletter
\def\maxwidth{ %
  \ifdim\Gin@nat@width>\linewidth
    \linewidth
  \else
    \Gin@nat@width
  \fi
}
\makeatother

\definecolor{fgcolor}{rgb}{0.2, 0.2, 0.2}
\newcommand{\hlnumber}[1]{\textcolor[rgb]{0,0,0}{#1}}%
\newcommand{\hlfunctioncall}[1]{\textcolor[rgb]{0.501960784313725,0,0.329411764705882}{\textbf{#1}}}%
\newcommand{\hlstring}[1]{\textcolor[rgb]{0.6,0.6,1}{#1}}%
\newcommand{\hlkeyword}[1]{\textcolor[rgb]{0,0,0}{\textbf{#1}}}%
\newcommand{\hlargument}[1]{\textcolor[rgb]{0.690196078431373,0.250980392156863,0.0196078431372549}{#1}}%
\newcommand{\hlcomment}[1]{\textcolor[rgb]{0.180392156862745,0.6,0.341176470588235}{#1}}%
\newcommand{\hlroxygencomment}[1]{\textcolor[rgb]{0.43921568627451,0.47843137254902,0.701960784313725}{#1}}%
\newcommand{\hlformalargs}[1]{\textcolor[rgb]{0.690196078431373,0.250980392156863,0.0196078431372549}{#1}}%
\newcommand{\hleqformalargs}[1]{\textcolor[rgb]{0.690196078431373,0.250980392156863,0.0196078431372549}{#1}}%
\newcommand{\hlassignement}[1]{\textcolor[rgb]{0,0,0}{\textbf{#1}}}%
\newcommand{\hlpackage}[1]{\textcolor[rgb]{0.588235294117647,0.709803921568627,0.145098039215686}{#1}}%
\newcommand{\hlslot}[1]{\textit{#1}}%
\newcommand{\hlsymbol}[1]{\textcolor[rgb]{0,0,0}{#1}}%
\newcommand{\hlprompt}[1]{\textcolor[rgb]{0.2,0.2,0.2}{#1}}%

\usepackage{framed}
\makeatletter
\newenvironment{kframe}{%
 \def\at@end@of@kframe{}%
 \ifinner\ifhmode%
  \def\at@end@of@kframe{\end{minipage}}%
  \begin{minipage}{\columnwidth}%
 \fi\fi%
 \def\FrameCommand##1{\hskip\@totalleftmargin \hskip-\fboxsep
 \colorbox{shadecolor}{##1}\hskip-\fboxsep
     % There is no \\@totalrightmargin, so:
     \hskip-\linewidth \hskip-\@totalleftmargin \hskip\columnwidth}%
 \MakeFramed {\advance\hsize-\width
   \@totalleftmargin\z@ \linewidth\hsize
   \@setminipage}}%
 {\par\unskip\endMakeFramed%
 \at@end@of@kframe}
\makeatother

\definecolor{shadecolor}{rgb}{.97, .97, .97}
\definecolor{messagecolor}{rgb}{0, 0, 0}
\definecolor{warningcolor}{rgb}{1, 0, 1}
\definecolor{errorcolor}{rgb}{1, 0, 0}
\newenvironment{knitrout}{}{} % an empty environment to be redefined in TeX

\usepackage{alltt}
\title{ggplot2 Gallery}
\IfFileExists{upquote.sty}{\usepackage{upquote}}{}
\begin{document}
\maketitle
\tableofcontents




% all geoms in ggplot2



\section{geom\_abline}

\begin{knitrout}
\definecolor{shadecolor}{rgb}{0.969, 0.969, 0.969}\color{fgcolor}\begin{kframe}
\begin{alltt}
\hlcomment{### Name: geom_abline}
\hlcomment{### Title: Line specified by slope and intercept.}
\hlcomment{### Aliases: geom_abline}

\hlcomment{### ** Examples}

p <- \hlfunctioncall{qplot}(wt, mpg, data = mtcars)

\hlcomment{# Fixed slopes and intercepts}
p + \hlfunctioncall{geom_abline}() \hlcomment{# Can't see it - outside the range of the data}
\end{alltt}
\end{kframe}
\includegraphics[width=\maxwidth]{figure/021-ggplot2-geoms-geom_abline1} 
\begin{kframe}\begin{alltt}
p + \hlfunctioncall{geom_abline}(intercept = 20)
\end{alltt}
\end{kframe}
\includegraphics[width=\maxwidth]{figure/021-ggplot2-geoms-geom_abline2} 
\begin{kframe}\begin{alltt}

\hlcomment{# Calculate slope and intercept of line of best fit}
\hlfunctioncall{coef}(\hlfunctioncall{lm}(mpg ~ wt, data = mtcars))
\end{alltt}
\begin{verbatim}
## (Intercept)          wt 
##      37.285      -5.344
\end{verbatim}
\begin{alltt}
p + \hlfunctioncall{geom_abline}(intercept = 37, slope = -5)
\end{alltt}
\end{kframe}
\includegraphics[width=\maxwidth]{figure/021-ggplot2-geoms-geom_abline3} 
\begin{kframe}\begin{alltt}
p + \hlfunctioncall{geom_abline}(intercept = 10, colour = \hlstring{"red"}, size = 2)
\end{alltt}
\end{kframe}
\includegraphics[width=\maxwidth]{figure/021-ggplot2-geoms-geom_abline4} 
\begin{kframe}\begin{alltt}

\hlcomment{# See ?stat_smooth for fitting smooth models to data}
p + \hlfunctioncall{stat_smooth}(method=\hlstring{"lm"}, se=FALSE)
\end{alltt}
\end{kframe}
\includegraphics[width=\maxwidth]{figure/021-ggplot2-geoms-geom_abline5} 
\begin{kframe}\begin{alltt}

\hlcomment{# Slopes and intercepts as data}
p <- \hlfunctioncall{ggplot}(mtcars, \hlfunctioncall{aes}(x = wt, y=mpg), . ~ cyl) + \hlfunctioncall{geom_point}()
df <- \hlfunctioncall{data.frame}(a=\hlfunctioncall{rnorm}(10, 25), b=\hlfunctioncall{rnorm}(10, 0))
p + \hlfunctioncall{geom_abline}(\hlfunctioncall{aes}(intercept=a, slope=b), data=df)
\end{alltt}
\end{kframe}
\includegraphics[width=\maxwidth]{figure/021-ggplot2-geoms-geom_abline6} 
\begin{kframe}\begin{alltt}

\hlcomment{# Slopes and intercepts from linear model}
\hlfunctioncall{library}(plyr)
coefs <- \hlfunctioncall{ddply}(mtcars, \hlfunctioncall{.}(cyl), \hlfunctioncall{function}(df) \{
  m <- \hlfunctioncall{lm}(mpg ~ wt, data=df)
  \hlfunctioncall{data.frame}(a = \hlfunctioncall{coef}(m)[1], b = \hlfunctioncall{coef}(m)[2])
\})
\hlfunctioncall{str}(coefs)
\end{alltt}
\begin{verbatim}
## 'data.frame':	3 obs. of  3 variables:
##  $ cyl: num  4 6 8
##  $ a  : num  39.6 28.4 23.9
##  $ b  : num  -5.65 -2.78 -2.19
\end{verbatim}
\begin{alltt}
p + \hlfunctioncall{geom_abline}(data=coefs, \hlfunctioncall{aes}(intercept=a, slope=b))
\end{alltt}
\end{kframe}
\includegraphics[width=\maxwidth]{figure/021-ggplot2-geoms-geom_abline7} 
\begin{kframe}\begin{alltt}

\hlcomment{# It's actually a bit easier to do this with stat_smooth}
p + \hlfunctioncall{geom_smooth}(\hlfunctioncall{aes}(group=cyl), method=\hlstring{"lm"})
\end{alltt}
\end{kframe}
\includegraphics[width=\maxwidth]{figure/021-ggplot2-geoms-geom_abline8} 
\begin{kframe}\begin{alltt}
p + \hlfunctioncall{geom_smooth}(\hlfunctioncall{aes}(group=cyl), method=\hlstring{"lm"}, fullrange=TRUE)
\end{alltt}
\end{kframe}
\includegraphics[width=\maxwidth]{figure/021-ggplot2-geoms-geom_abline9} 
\begin{kframe}\begin{alltt}

\hlcomment{# With coordinate transforms}
p + \hlfunctioncall{geom_abline}(intercept = 37, slope = -5) + \hlfunctioncall{coord_flip}()
\end{alltt}
\end{kframe}
\includegraphics[width=\maxwidth]{figure/021-ggplot2-geoms-geom_abline10} 
\begin{kframe}\begin{alltt}
p + \hlfunctioncall{geom_abline}(intercept = 37, slope = -5) + \hlfunctioncall{coord_polar}()
\end{alltt}
\end{kframe}
\includegraphics[width=\maxwidth]{figure/021-ggplot2-geoms-geom_abline11} 
\begin{kframe}\begin{alltt}


\end{alltt}
\end{kframe}
\end{knitrout}



\section{geom\_area}

\begin{knitrout}
\definecolor{shadecolor}{rgb}{0.969, 0.969, 0.969}\color{fgcolor}\begin{kframe}
\begin{alltt}
\hlcomment{### Name: geom_area}
\hlcomment{### Title: Area plot.}
\hlcomment{### Aliases: geom_area}

\hlcomment{### ** Examples}

\hlcomment{# see geom_ribbon}



\end{alltt}
\end{kframe}
\end{knitrout}



\section{geom\_bar}

\begin{knitrout}
\definecolor{shadecolor}{rgb}{0.969, 0.969, 0.969}\color{fgcolor}\begin{kframe}
\begin{alltt}
\hlcomment{### Name: geom_bar}
\hlcomment{### Title: Bars, rectangles with bases on x-axis}
\hlcomment{### Aliases: geom_bar}

\hlcomment{### ** Examples}

\hlcomment{## No test: }
\hlcomment{# Generate data}
c <- \hlfunctioncall{ggplot}(mtcars, \hlfunctioncall{aes}(\hlfunctioncall{factor}(cyl)))

c + \hlfunctioncall{geom_bar}()
\end{alltt}
\end{kframe}
\includegraphics[width=\maxwidth]{figure/021-ggplot2-geoms-geom_bar1} 
\begin{kframe}\begin{alltt}
c + \hlfunctioncall{geom_bar}(width=.5)
\end{alltt}
\end{kframe}
\includegraphics[width=\maxwidth]{figure/021-ggplot2-geoms-geom_bar2} 
\begin{kframe}\begin{alltt}
c + \hlfunctioncall{geom_bar}() + \hlfunctioncall{coord_flip}()
\end{alltt}
\end{kframe}
\includegraphics[width=\maxwidth]{figure/021-ggplot2-geoms-geom_bar3} 
\begin{kframe}\begin{alltt}
c + \hlfunctioncall{geom_bar}(fill=\hlstring{"white"}, colour=\hlstring{"darkgreen"})
\end{alltt}
\end{kframe}
\includegraphics[width=\maxwidth]{figure/021-ggplot2-geoms-geom_bar4} 
\begin{kframe}\begin{alltt}

\hlcomment{# Use qplot}
\hlfunctioncall{qplot}(\hlfunctioncall{factor}(cyl), data=mtcars, geom=\hlstring{"bar"})
\end{alltt}
\end{kframe}
\includegraphics[width=\maxwidth]{figure/021-ggplot2-geoms-geom_bar5} 
\begin{kframe}\begin{alltt}
\hlfunctioncall{qplot}(\hlfunctioncall{factor}(cyl), data=mtcars, geom=\hlstring{"bar"}, fill=\hlfunctioncall{factor}(cyl))
\end{alltt}
\end{kframe}
\includegraphics[width=\maxwidth]{figure/021-ggplot2-geoms-geom_bar6} 
\begin{kframe}\begin{alltt}

\hlcomment{# Stacked bar charts}
\hlfunctioncall{qplot}(\hlfunctioncall{factor}(cyl), data=mtcars, geom=\hlstring{"bar"}, fill=\hlfunctioncall{factor}(vs))
\end{alltt}
\end{kframe}
\includegraphics[width=\maxwidth]{figure/021-ggplot2-geoms-geom_bar7} 
\begin{kframe}\begin{alltt}
\hlfunctioncall{qplot}(\hlfunctioncall{factor}(cyl), data=mtcars, geom=\hlstring{"bar"}, fill=\hlfunctioncall{factor}(gear))
\end{alltt}
\end{kframe}
\includegraphics[width=\maxwidth]{figure/021-ggplot2-geoms-geom_bar8} 
\begin{kframe}\begin{alltt}

\hlcomment{# Stacked bar charts are easy in ggplot2, but not effective visually,}
\hlcomment{# particularly when there are many different things being stacked}
\hlfunctioncall{ggplot}(diamonds, \hlfunctioncall{aes}(clarity, fill=cut)) + \hlfunctioncall{geom_bar}()
\end{alltt}
\end{kframe}
\includegraphics[width=\maxwidth]{figure/021-ggplot2-geoms-geom_bar9} 
\begin{kframe}\begin{alltt}
\hlfunctioncall{ggplot}(diamonds, \hlfunctioncall{aes}(color, fill=cut)) + \hlfunctioncall{geom_bar}() + \hlfunctioncall{coord_flip}()
\end{alltt}
\end{kframe}
\includegraphics[width=\maxwidth]{figure/021-ggplot2-geoms-geom_bar10} 
\begin{kframe}\begin{alltt}

\hlcomment{# Faceting is a good alternative:}
\hlfunctioncall{ggplot}(diamonds, \hlfunctioncall{aes}(clarity)) + \hlfunctioncall{geom_bar}() +
  \hlfunctioncall{facet_wrap}(~ cut)
\end{alltt}
\end{kframe}
\includegraphics[width=\maxwidth]{figure/021-ggplot2-geoms-geom_bar11} 
\begin{kframe}\begin{alltt}
\hlcomment{# If the x axis is ordered, using a line instead of bars is another}
\hlcomment{# possibility:}
\hlfunctioncall{ggplot}(diamonds, \hlfunctioncall{aes}(clarity)) +
  \hlfunctioncall{geom_freqpoly}(\hlfunctioncall{aes}(group = cut, colour = cut))
\end{alltt}
\end{kframe}
\includegraphics[width=\maxwidth]{figure/021-ggplot2-geoms-geom_bar12} 
\begin{kframe}\begin{alltt}

\hlcomment{# Dodged bar charts}
\hlfunctioncall{ggplot}(diamonds, \hlfunctioncall{aes}(clarity, fill=cut)) + \hlfunctioncall{geom_bar}(position=\hlstring{"dodge"})
\end{alltt}
\end{kframe}
\includegraphics[width=\maxwidth]{figure/021-ggplot2-geoms-geom_bar13} 
\begin{kframe}\begin{alltt}
\hlcomment{# compare with}
\hlfunctioncall{ggplot}(diamonds, \hlfunctioncall{aes}(cut, fill=cut)) + \hlfunctioncall{geom_bar}() +
  \hlfunctioncall{facet_grid}(. ~ clarity)
\end{alltt}
\end{kframe}
\includegraphics[width=\maxwidth]{figure/021-ggplot2-geoms-geom_bar14} 
\begin{kframe}\begin{alltt}

\hlcomment{# But again, probably better to use frequency polygons instead:}
\hlfunctioncall{ggplot}(diamonds, \hlfunctioncall{aes}(clarity, colour=cut)) +
  \hlfunctioncall{geom_freqpoly}(\hlfunctioncall{aes}(group = cut))
\end{alltt}
\end{kframe}
\includegraphics[width=\maxwidth]{figure/021-ggplot2-geoms-geom_bar15} 
\begin{kframe}\begin{alltt}

\hlcomment{# Often we don't want the height of the bar to represent the}
\hlcomment{# count of observations, but the sum of some other variable.}
\hlcomment{# For example, the following plot shows the number of diamonds}
\hlcomment{# of each colour}
\hlfunctioncall{qplot}(color, data=diamonds, geom=\hlstring{"bar"})
\end{alltt}
\end{kframe}
\includegraphics[width=\maxwidth]{figure/021-ggplot2-geoms-geom_bar16} 
\begin{kframe}\begin{alltt}
\hlcomment{# If, however, we want to see the total number of carats in each colour}
\hlcomment{# we need to weight by the carat variable}
\hlfunctioncall{qplot}(color, data=diamonds, geom=\hlstring{"bar"}, weight=carat, ylab=\hlstring{"carat"})
\end{alltt}
\end{kframe}
\includegraphics[width=\maxwidth]{figure/021-ggplot2-geoms-geom_bar17} 
\begin{kframe}\begin{alltt}

\hlcomment{# A bar chart used to display means}
meanprice <- \hlfunctioncall{tapply}(diamonds$price, diamonds$cut, mean)
cut <- \hlfunctioncall{factor}(\hlfunctioncall{levels}(diamonds$cut), levels = \hlfunctioncall{levels}(diamonds$cut))
\hlfunctioncall{qplot}(cut, meanprice)
\end{alltt}
\end{kframe}
\includegraphics[width=\maxwidth]{figure/021-ggplot2-geoms-geom_bar18} 
\begin{kframe}\begin{alltt}
\hlfunctioncall{qplot}(cut, meanprice, geom=\hlstring{"bar"}, stat=\hlstring{"identity"})
\end{alltt}
\end{kframe}
\includegraphics[width=\maxwidth]{figure/021-ggplot2-geoms-geom_bar19} 
\begin{kframe}\begin{alltt}
\hlfunctioncall{qplot}(cut, meanprice, geom=\hlstring{"bar"}, stat=\hlstring{"identity"}, fill = \hlfunctioncall{I}(\hlstring{"grey50"}))
\end{alltt}
\end{kframe}
\includegraphics[width=\maxwidth]{figure/021-ggplot2-geoms-geom_bar20} 
\begin{kframe}\begin{alltt}

\hlcomment{# Another stacked bar chart example}
k <- \hlfunctioncall{ggplot}(mpg, \hlfunctioncall{aes}(manufacturer, fill=class))
k + \hlfunctioncall{geom_bar}()
\end{alltt}
\end{kframe}
\includegraphics[width=\maxwidth]{figure/021-ggplot2-geoms-geom_bar21} 
\begin{kframe}\begin{alltt}
\hlcomment{# Use scales to change aesthetics defaults}
k + \hlfunctioncall{geom_bar}() + \hlfunctioncall{scale_fill_brewer}()
\end{alltt}
\end{kframe}
\includegraphics[width=\maxwidth]{figure/021-ggplot2-geoms-geom_bar22} 
\begin{kframe}\begin{alltt}
k + \hlfunctioncall{geom_bar}() + \hlfunctioncall{scale_fill_grey}()
\end{alltt}
\end{kframe}
\includegraphics[width=\maxwidth]{figure/021-ggplot2-geoms-geom_bar23} 
\begin{kframe}\begin{alltt}

\hlcomment{# To change plot order of class varible}
\hlcomment{# use factor() to change order of levels}
mpg$class <- \hlfunctioncall{factor}(mpg$class, levels = \hlfunctioncall{c}(\hlstring{"midsize"}, \hlstring{"minivan"},
\hlstring{"suv"}, \hlstring{"compact"}, \hlstring{"2seater"}, \hlstring{"subcompact"}, \hlstring{"pickup"}))
m <- \hlfunctioncall{ggplot}(mpg, \hlfunctioncall{aes}(manufacturer, fill=class))
m + \hlfunctioncall{geom_bar}()
\end{alltt}
\end{kframe}
\includegraphics[width=\maxwidth]{figure/021-ggplot2-geoms-geom_bar24} 
\begin{kframe}\begin{alltt}
\hlcomment{## End(No test)}


\end{alltt}
\end{kframe}
\end{knitrout}



\section{geom\_bin2d}

\begin{knitrout}
\definecolor{shadecolor}{rgb}{0.969, 0.969, 0.969}\color{fgcolor}\begin{kframe}
\begin{alltt}
\hlcomment{### Name: geom_bin2d}
\hlcomment{### Title: Add heatmap of 2d bin counts.}
\hlcomment{### Aliases: geom_bin2d}

\hlcomment{### ** Examples}

d <- \hlfunctioncall{ggplot}(diamonds, \hlfunctioncall{aes}(x = x, y = y)) + \hlfunctioncall{xlim}(4,10) + \hlfunctioncall{ylim}(4,10)
d + \hlfunctioncall{geom_bin2d}()
\end{alltt}
\end{kframe}
\includegraphics[width=\maxwidth]{figure/021-ggplot2-geoms-geom_bin2d1} 
\begin{kframe}\begin{alltt}
d + \hlfunctioncall{geom_bin2d}(binwidth = \hlfunctioncall{c}(0.1, 0.1))
\end{alltt}
\end{kframe}
\includegraphics[width=\maxwidth]{figure/021-ggplot2-geoms-geom_bin2d2} 
\begin{kframe}\begin{alltt}

\hlcomment{# See ?stat_bin2d for more examples}


\end{alltt}
\end{kframe}
\end{knitrout}



\section{geom\_blank}

\begin{knitrout}
\definecolor{shadecolor}{rgb}{0.969, 0.969, 0.969}\color{fgcolor}\begin{kframe}
\begin{alltt}
\hlcomment{### Name: geom_blank}
\hlcomment{### Title: Blank, draws nothing.}
\hlcomment{### Aliases: geom_blank}

\hlcomment{### ** Examples}

\hlfunctioncall{qplot}(length, rating, data = movies, geom = \hlstring{"blank"})
\end{alltt}
\end{kframe}
\includegraphics[width=\maxwidth]{figure/021-ggplot2-geoms-geom_blank1} 
\begin{kframe}\begin{alltt}
\hlcomment{# Nothing to see here!}

\hlcomment{# Take the following scatter plot}
a <- \hlfunctioncall{ggplot}(mtcars, \hlfunctioncall{aes}(x = wt, y = mpg), . ~ cyl) + \hlfunctioncall{geom_point}()
\hlcomment{# Add to that some lines with geom_abline()}
df <- \hlfunctioncall{data.frame}(a = \hlfunctioncall{rnorm}(10, 25), b = \hlfunctioncall{rnorm}(10, 0))
a + \hlfunctioncall{geom_abline}(\hlfunctioncall{aes}(intercept = a, slope = b), data = df)
\end{alltt}
\end{kframe}
\includegraphics[width=\maxwidth]{figure/021-ggplot2-geoms-geom_blank2} 
\begin{kframe}\begin{alltt}
\hlcomment{# Suppose you then wanted to remove the geom_point layer}
\hlcomment{# If you just remove geom_point, you will get an error}
b <- \hlfunctioncall{ggplot}(mtcars, \hlfunctioncall{aes}(x = wt, y = mpg))
\hlcomment{## Not run: b + geom_abline(aes(intercept = a, slope = b), data = df)}
\hlcomment{# Switching to geom_blank() gets the desired plot}
c <- \hlfunctioncall{ggplot}(mtcars, \hlfunctioncall{aes}(x = wt, y = mpg)) + \hlfunctioncall{geom_blank}()
c + \hlfunctioncall{geom_abline}(\hlfunctioncall{aes}(intercept = a, slope = b), data = df)
\end{alltt}
\end{kframe}
\includegraphics[width=\maxwidth]{figure/021-ggplot2-geoms-geom_blank3} 
\begin{kframe}\begin{alltt}


\end{alltt}
\end{kframe}
\end{knitrout}



\section{geom\_boxplot}

\begin{knitrout}
\definecolor{shadecolor}{rgb}{0.969, 0.969, 0.969}\color{fgcolor}\begin{kframe}
\begin{alltt}
\hlcomment{### Name: geom_boxplot}
\hlcomment{### Title: Box and whiskers plot.}
\hlcomment{### Aliases: geom_boxplot}

\hlcomment{### ** Examples}

\hlcomment{## No test: }
p <- \hlfunctioncall{ggplot}(mtcars, \hlfunctioncall{aes}(\hlfunctioncall{factor}(cyl), mpg))

p + \hlfunctioncall{geom_boxplot}()
\end{alltt}
\end{kframe}
\includegraphics[width=\maxwidth]{figure/021-ggplot2-geoms-geom_boxplot1} 
\begin{kframe}\begin{alltt}
\hlfunctioncall{qplot}(\hlfunctioncall{factor}(cyl), mpg, data = mtcars, geom = \hlstring{"boxplot"})
\end{alltt}
\end{kframe}
\includegraphics[width=\maxwidth]{figure/021-ggplot2-geoms-geom_boxplot2} 
\begin{kframe}\begin{alltt}

p + \hlfunctioncall{geom_boxplot}() + \hlfunctioncall{geom_jitter}()
\end{alltt}
\end{kframe}
\includegraphics[width=\maxwidth]{figure/021-ggplot2-geoms-geom_boxplot3} 
\begin{kframe}\begin{alltt}
p + \hlfunctioncall{geom_boxplot}() + \hlfunctioncall{coord_flip}()
\end{alltt}
\end{kframe}
\includegraphics[width=\maxwidth]{figure/021-ggplot2-geoms-geom_boxplot4} 
\begin{kframe}\begin{alltt}
\hlfunctioncall{qplot}(\hlfunctioncall{factor}(cyl), mpg, data = mtcars, geom = \hlstring{"boxplot"}) +
  \hlfunctioncall{coord_flip}()
\end{alltt}
\end{kframe}
\includegraphics[width=\maxwidth]{figure/021-ggplot2-geoms-geom_boxplot5} 
\begin{kframe}\begin{alltt}

p + \hlfunctioncall{geom_boxplot}(notch = TRUE)
\end{alltt}


{\ttfamily\noindent\textcolor{warningcolor}{\#\# Warning: notch went outside hinges. Try setting notch=FALSE.}}

{\ttfamily\noindent\textcolor{warningcolor}{\#\# Warning: notch went outside hinges. Try setting notch=FALSE.}}\end{kframe}
\includegraphics[width=\maxwidth]{figure/021-ggplot2-geoms-geom_boxplot6} 
\begin{kframe}\begin{alltt}
p + \hlfunctioncall{geom_boxplot}(notch = TRUE, notchwidth = .3)
\end{alltt}


{\ttfamily\noindent\textcolor{warningcolor}{\#\# Warning: notch went outside hinges. Try setting notch=FALSE.}}

{\ttfamily\noindent\textcolor{warningcolor}{\#\# Warning: notch went outside hinges. Try setting notch=FALSE.}}\end{kframe}
\includegraphics[width=\maxwidth]{figure/021-ggplot2-geoms-geom_boxplot7} 
\begin{kframe}\begin{alltt}

p + \hlfunctioncall{geom_boxplot}(outlier.colour = \hlstring{"green"}, outlier.size = 3)
\end{alltt}
\end{kframe}
\includegraphics[width=\maxwidth]{figure/021-ggplot2-geoms-geom_boxplot8} 
\begin{kframe}\begin{alltt}

\hlcomment{# Add aesthetic mappings}
\hlcomment{# Note that boxplots are automatically dodged when any aesthetic is}
\hlcomment{# a factor}
p + \hlfunctioncall{geom_boxplot}(\hlfunctioncall{aes}(fill = cyl))
\end{alltt}
\end{kframe}
\includegraphics[width=\maxwidth]{figure/021-ggplot2-geoms-geom_boxplot9} 
\begin{kframe}\begin{alltt}
p + \hlfunctioncall{geom_boxplot}(\hlfunctioncall{aes}(fill = \hlfunctioncall{factor}(cyl)))
\end{alltt}
\end{kframe}
\includegraphics[width=\maxwidth]{figure/021-ggplot2-geoms-geom_boxplot10} 
\begin{kframe}\begin{alltt}
p + \hlfunctioncall{geom_boxplot}(\hlfunctioncall{aes}(fill = \hlfunctioncall{factor}(vs)))
\end{alltt}
\end{kframe}
\includegraphics[width=\maxwidth]{figure/021-ggplot2-geoms-geom_boxplot11} 
\begin{kframe}\begin{alltt}
p + \hlfunctioncall{geom_boxplot}(\hlfunctioncall{aes}(fill = \hlfunctioncall{factor}(am)))
\end{alltt}
\end{kframe}
\includegraphics[width=\maxwidth]{figure/021-ggplot2-geoms-geom_boxplot12} 
\begin{kframe}\begin{alltt}

\hlcomment{# Set aesthetics to fixed value}
p + \hlfunctioncall{geom_boxplot}(fill = \hlstring{"grey80"}, colour = \hlstring{"#3366FF"})
\end{alltt}
\end{kframe}
\includegraphics[width=\maxwidth]{figure/021-ggplot2-geoms-geom_boxplot13} 
\begin{kframe}\begin{alltt}
\hlfunctioncall{qplot}(\hlfunctioncall{factor}(cyl), mpg, data = mtcars, geom = \hlstring{"boxplot"},
  colour = \hlfunctioncall{I}(\hlstring{"#3366FF"}))
\end{alltt}
\end{kframe}
\includegraphics[width=\maxwidth]{figure/021-ggplot2-geoms-geom_boxplot14} 
\begin{kframe}\begin{alltt}

\hlcomment{# Scales vs. coordinate transforms -------}
\hlcomment{# Scale transformations occur before the boxplot statistics are computed.}
\hlcomment{# Coordinate transformations occur afterwards.  Observe the effect on the}
\hlcomment{# number of outliers.}
\hlfunctioncall{library}(plyr) \hlcomment{# to access round_any}
m <- \hlfunctioncall{ggplot}(movies, \hlfunctioncall{aes}(y = votes, x = rating,
   group = \hlfunctioncall{round_any}(rating, 0.5)))
m + \hlfunctioncall{geom_boxplot}()
\end{alltt}


{\ttfamily\noindent\textcolor{warningcolor}{\#\# Warning: position\_dodge requires constant width: output may be incorrect}}\end{kframe}
\includegraphics[width=\maxwidth]{figure/021-ggplot2-geoms-geom_boxplot15} 
\begin{kframe}\begin{alltt}
m + \hlfunctioncall{geom_boxplot}() + \hlfunctioncall{scale_y_log10}()
\end{alltt}


{\ttfamily\noindent\textcolor{warningcolor}{\#\# Warning: position\_dodge requires constant width: output may be incorrect}}\end{kframe}
\includegraphics[width=\maxwidth]{figure/021-ggplot2-geoms-geom_boxplot16} 
\begin{kframe}\begin{alltt}
m + \hlfunctioncall{geom_boxplot}() + \hlfunctioncall{coord_trans}(y = \hlstring{"log10"})
\end{alltt}


{\ttfamily\noindent\textcolor{warningcolor}{\#\# Warning: position\_dodge requires constant width: output may be incorrect}}\end{kframe}
\includegraphics[width=\maxwidth]{figure/021-ggplot2-geoms-geom_boxplot17} 
\begin{kframe}\begin{alltt}
m + \hlfunctioncall{geom_boxplot}() + \hlfunctioncall{scale_y_log10}() + \hlfunctioncall{coord_trans}(y = \hlstring{"log10"})
\end{alltt}


{\ttfamily\noindent\textcolor{warningcolor}{\#\# Warning: position\_dodge requires constant width: output may be incorrect}}\end{kframe}
\includegraphics[width=\maxwidth]{figure/021-ggplot2-geoms-geom_boxplot18} 
\begin{kframe}\begin{alltt}

\hlcomment{# Boxplots with continuous x:}
\hlcomment{# Use the group aesthetic to group observations in boxplots}
\hlfunctioncall{qplot}(year, budget, data = movies, geom = \hlstring{"boxplot"})
\end{alltt}


{\ttfamily\noindent\textcolor{warningcolor}{\#\# Warning: Removed 53573 rows containing non-finite values (stat\_boxplot).}}\end{kframe}
\includegraphics[width=\maxwidth]{figure/021-ggplot2-geoms-geom_boxplot19} 
\begin{kframe}\begin{alltt}
\hlfunctioncall{qplot}(year, budget, data = movies, geom = \hlstring{"boxplot"},
  group = \hlfunctioncall{round_any}(year, 10, floor))
\end{alltt}


{\ttfamily\noindent\textcolor{warningcolor}{\#\# Warning: Removed 53573 rows containing non-finite values (stat\_boxplot).}}

{\ttfamily\noindent\textcolor{warningcolor}{\#\# Warning: position\_dodge requires constant width: output may be incorrect}}\end{kframe}
\includegraphics[width=\maxwidth]{figure/021-ggplot2-geoms-geom_boxplot20} 
\begin{kframe}\begin{alltt}

\hlcomment{# Using precomputed statistics}
\hlcomment{# generate sample data}
abc <- \hlfunctioncall{adply}(\hlfunctioncall{matrix}(\hlfunctioncall{rnorm}(100), ncol = 5), 2, quantile, \hlfunctioncall{c}(0, .25, .5, .75, 1))
b <- \hlfunctioncall{ggplot}(abc, \hlfunctioncall{aes}(x = X1, ymin = `0%`, lower = `25%`, middle = `50%`, upper = `75%`, ymax = `100%`))
b + \hlfunctioncall{geom_boxplot}(stat = \hlstring{"identity"})
\end{alltt}
\end{kframe}
\includegraphics[width=\maxwidth]{figure/021-ggplot2-geoms-geom_boxplot21} 
\begin{kframe}\begin{alltt}
b + \hlfunctioncall{geom_boxplot}(stat = \hlstring{"identity"}) + \hlfunctioncall{coord_flip}()
\end{alltt}
\end{kframe}
\includegraphics[width=\maxwidth]{figure/021-ggplot2-geoms-geom_boxplot22} 
\begin{kframe}\begin{alltt}
b + \hlfunctioncall{geom_boxplot}(\hlfunctioncall{aes}(fill = X1), stat = \hlstring{"identity"})
\end{alltt}
\end{kframe}
\includegraphics[width=\maxwidth]{figure/021-ggplot2-geoms-geom_boxplot23} 
\begin{kframe}\begin{alltt}
\hlcomment{## End(No test)}


\end{alltt}
\end{kframe}
\end{knitrout}



\section{geom\_contour}

\begin{knitrout}
\definecolor{shadecolor}{rgb}{0.969, 0.969, 0.969}\color{fgcolor}\begin{kframe}
\begin{alltt}
\hlcomment{### Name: geom_contour}
\hlcomment{### Title: Display contours of a 3d surface in 2d.}
\hlcomment{### Aliases: geom_contour}

\hlcomment{### ** Examples}

\hlcomment{# See stat_contour for examples}



\end{alltt}
\end{kframe}
\end{knitrout}



\section{geom\_crossbar}

\begin{knitrout}
\definecolor{shadecolor}{rgb}{0.969, 0.969, 0.969}\color{fgcolor}\begin{kframe}
\begin{alltt}
\hlcomment{### Name: geom_crossbar}
\hlcomment{### Title: Hollow bar with middle indicated by horizontal line.}
\hlcomment{### Aliases: geom_crossbar}

\hlcomment{### ** Examples}

\hlcomment{# See geom_linerange for examples}



\end{alltt}
\end{kframe}
\end{knitrout}



\section{geom\_density}

\begin{knitrout}
\definecolor{shadecolor}{rgb}{0.969, 0.969, 0.969}\color{fgcolor}\begin{kframe}
\begin{alltt}
\hlcomment{### Name: geom_density}
\hlcomment{### Title: Display a smooth density estimate.}
\hlcomment{### Aliases: geom_density}

\hlcomment{### ** Examples}

\hlcomment{# See stat_density for examples}



\end{alltt}
\end{kframe}
\end{knitrout}



\section{geom\_density2d}

\begin{knitrout}
\definecolor{shadecolor}{rgb}{0.969, 0.969, 0.969}\color{fgcolor}\begin{kframe}
\begin{alltt}
\hlcomment{### Name: geom_density2d}
\hlcomment{### Title: Contours from a 2d density estimate.}
\hlcomment{### Aliases: geom_density2d}

\hlcomment{### ** Examples}

\hlcomment{# See stat_density2d for examples}



\end{alltt}
\end{kframe}
\end{knitrout}



\section{geom\_dotplot}

\begin{knitrout}
\definecolor{shadecolor}{rgb}{0.969, 0.969, 0.969}\color{fgcolor}\begin{kframe}
\begin{alltt}
\hlcomment{### Name: geom_dotplot}
\hlcomment{### Title: Dot plot}
\hlcomment{### Aliases: geom_dotplot}

\hlcomment{### ** Examples}

\hlfunctioncall{ggplot}(mtcars, \hlfunctioncall{aes}(x = mpg)) + \hlfunctioncall{geom_dotplot}()
\end{alltt}


{\ttfamily\noindent\itshape\textcolor{messagecolor}{\#\# stat\_bindot: binwidth defaulted to range/30. Use 'binwidth = x' to adjust this.}}\end{kframe}
\includegraphics[width=\maxwidth]{figure/021-ggplot2-geoms-geom_dotplot1} 
\begin{kframe}\begin{alltt}
\hlfunctioncall{ggplot}(mtcars, \hlfunctioncall{aes}(x = mpg)) + \hlfunctioncall{geom_dotplot}(binwidth = 1.5)
\end{alltt}
\end{kframe}
\includegraphics[width=\maxwidth]{figure/021-ggplot2-geoms-geom_dotplot2} 
\begin{kframe}\begin{alltt}

\hlcomment{# Use fixed-width bins}
\hlfunctioncall{ggplot}(mtcars, \hlfunctioncall{aes}(x = mpg)) +
  \hlfunctioncall{geom_dotplot}(method=\hlstring{"histodot"}, binwidth = 1.5)
\end{alltt}
\end{kframe}
\includegraphics[width=\maxwidth]{figure/021-ggplot2-geoms-geom_dotplot3} 
\begin{kframe}\begin{alltt}

\hlcomment{# Some other stacking methods}
\hlfunctioncall{ggplot}(mtcars, \hlfunctioncall{aes}(x = mpg)) +
  \hlfunctioncall{geom_dotplot}(binwidth = 1.5, stackdir = \hlstring{"center"})
\end{alltt}
\end{kframe}
\includegraphics[width=\maxwidth]{figure/021-ggplot2-geoms-geom_dotplot4} 
\begin{kframe}\begin{alltt}
\hlfunctioncall{ggplot}(mtcars, \hlfunctioncall{aes}(x = mpg)) +
  \hlfunctioncall{geom_dotplot}(binwidth = 1.5, stackdir = \hlstring{"centerwhole"})
\end{alltt}
\end{kframe}
\includegraphics[width=\maxwidth]{figure/021-ggplot2-geoms-geom_dotplot5} 
\begin{kframe}\begin{alltt}

\hlcomment{# y axis isn't really meaningful, so hide it}
\hlfunctioncall{ggplot}(mtcars, \hlfunctioncall{aes}(x = mpg)) + \hlfunctioncall{geom_dotplot}(binwidth = 1.5) +
  \hlfunctioncall{scale_y_continuous}(name = \hlstring{""}, breaks = NA)
\end{alltt}


{\ttfamily\noindent\textcolor{warningcolor}{\#\# Warning: breaks = NA is deprecated. Please use breaks = NULL to remove breaks in the scale.}}\end{kframe}
\includegraphics[width=\maxwidth]{figure/021-ggplot2-geoms-geom_dotplot6} 
\begin{kframe}\begin{alltt}

\hlcomment{# Overlap dots vertically}
\hlfunctioncall{ggplot}(mtcars, \hlfunctioncall{aes}(x = mpg)) + \hlfunctioncall{geom_dotplot}(binwidth = 1.5, stackratio = .7)
\end{alltt}
\end{kframe}
\includegraphics[width=\maxwidth]{figure/021-ggplot2-geoms-geom_dotplot7} 
\begin{kframe}\begin{alltt}

\hlcomment{# Expand dot diameter}
\hlfunctioncall{ggplot}(mtcars, \hlfunctioncall{aes}(x  =mpg)) + \hlfunctioncall{geom_dotplot}(binwidth = 1.5, dotsize = 1.25)
\end{alltt}
\end{kframe}
\includegraphics[width=\maxwidth]{figure/021-ggplot2-geoms-geom_dotplot8} 
\begin{kframe}\begin{alltt}


\hlcomment{# Examples with stacking along y axis instead of x}
\hlfunctioncall{ggplot}(mtcars, \hlfunctioncall{aes}(x = 1, y = mpg)) +
  \hlfunctioncall{geom_dotplot}(binaxis = \hlstring{"y"}, stackdir = \hlstring{"center"})
\end{alltt}


{\ttfamily\noindent\itshape\textcolor{messagecolor}{\#\# stat\_bindot: binwidth defaulted to range/30. Use 'binwidth = x' to adjust this.}}\end{kframe}
\includegraphics[width=\maxwidth]{figure/021-ggplot2-geoms-geom_dotplot9} 
\begin{kframe}\begin{alltt}

\hlfunctioncall{ggplot}(mtcars, \hlfunctioncall{aes}(x = \hlfunctioncall{factor}(cyl), y = mpg)) +
  \hlfunctioncall{geom_dotplot}(binaxis = \hlstring{"y"}, stackdir = \hlstring{"center"})
\end{alltt}


{\ttfamily\noindent\itshape\textcolor{messagecolor}{\#\# stat\_bindot: binwidth defaulted to range/30. Use 'binwidth = x' to adjust this.}}\end{kframe}
\includegraphics[width=\maxwidth]{figure/021-ggplot2-geoms-geom_dotplot10} 
\begin{kframe}\begin{alltt}

\hlfunctioncall{ggplot}(mtcars, \hlfunctioncall{aes}(x = \hlfunctioncall{factor}(cyl), y = mpg)) +
  \hlfunctioncall{geom_dotplot}(binaxis = \hlstring{"y"}, stackdir = \hlstring{"centerwhole"})
\end{alltt}


{\ttfamily\noindent\itshape\textcolor{messagecolor}{\#\# stat\_bindot: binwidth defaulted to range/30. Use 'binwidth = x' to adjust this.}}\end{kframe}
\includegraphics[width=\maxwidth]{figure/021-ggplot2-geoms-geom_dotplot11} 
\begin{kframe}\begin{alltt}

\hlfunctioncall{ggplot}(mtcars, \hlfunctioncall{aes}(x = \hlfunctioncall{factor}(vs), fill = \hlfunctioncall{factor}(cyl), y = mpg)) +
  \hlfunctioncall{geom_dotplot}(binaxis = \hlstring{"y"}, stackdir = \hlstring{"center"}, position = \hlstring{"dodge"})
\end{alltt}


{\ttfamily\noindent\itshape\textcolor{messagecolor}{\#\# stat\_bindot: binwidth defaulted to range/30. Use 'binwidth = x' to adjust this.}}\end{kframe}
\includegraphics[width=\maxwidth]{figure/021-ggplot2-geoms-geom_dotplot12} 
\begin{kframe}\begin{alltt}

\hlcomment{# binpositions="all" ensures that the bins are aligned between groups}
\hlfunctioncall{ggplot}(mtcars, \hlfunctioncall{aes}(x = \hlfunctioncall{factor}(am), y = mpg)) +
  \hlfunctioncall{geom_dotplot}(binaxis = \hlstring{"y"}, stackdir = \hlstring{"center"}, binpositions=\hlstring{"all"})
\end{alltt}


{\ttfamily\noindent\itshape\textcolor{messagecolor}{\#\# stat\_bindot: binwidth defaulted to range/30. Use 'binwidth = x' to adjust this.}}\end{kframe}
\includegraphics[width=\maxwidth]{figure/021-ggplot2-geoms-geom_dotplot13} 
\begin{kframe}\begin{alltt}

\hlcomment{# Stacking multiple groups, with different fill}
\hlfunctioncall{ggplot}(mtcars, \hlfunctioncall{aes}(x = mpg, fill = \hlfunctioncall{factor}(cyl))) +
  \hlfunctioncall{geom_dotplot}(stackgroups = TRUE, binwidth = 1, binpositions = \hlstring{"all"})
\end{alltt}
\end{kframe}
\includegraphics[width=\maxwidth]{figure/021-ggplot2-geoms-geom_dotplot14} 
\begin{kframe}\begin{alltt}

\hlfunctioncall{ggplot}(mtcars, \hlfunctioncall{aes}(x = mpg, fill = \hlfunctioncall{factor}(cyl))) +
  \hlfunctioncall{geom_dotplot}(stackgroups = TRUE, binwidth = 1, method = \hlstring{"histodot"})
\end{alltt}
\end{kframe}
\includegraphics[width=\maxwidth]{figure/021-ggplot2-geoms-geom_dotplot15} 
\begin{kframe}\begin{alltt}

\hlfunctioncall{ggplot}(mtcars, \hlfunctioncall{aes}(x = 1, y = mpg, fill = \hlfunctioncall{factor}(cyl))) +
  \hlfunctioncall{geom_dotplot}(binaxis = \hlstring{"y"}, stackgroups = TRUE, binwidth = 1, method = \hlstring{"histodot"})
\end{alltt}
\end{kframe}
\includegraphics[width=\maxwidth]{figure/021-ggplot2-geoms-geom_dotplot16} 
\begin{kframe}\begin{alltt}


\end{alltt}
\end{kframe}
\end{knitrout}



\section{geom\_errorbar}

\begin{knitrout}
\definecolor{shadecolor}{rgb}{0.969, 0.969, 0.969}\color{fgcolor}\begin{kframe}
\begin{alltt}
\hlcomment{### Name: geom_errorbar}
\hlcomment{### Title: Error bars.}
\hlcomment{### Aliases: geom_errorbar}

\hlcomment{### ** Examples}

\hlcomment{# Create a simple example dataset}
df <- \hlfunctioncall{data.frame}(
  trt = \hlfunctioncall{factor}(\hlfunctioncall{c}(1, 1, 2, 2)),
  resp = \hlfunctioncall{c}(1, 5, 3, 4),
  group = \hlfunctioncall{factor}(\hlfunctioncall{c}(1, 2, 1, 2)),
  se = \hlfunctioncall{c}(0.1, 0.3, 0.3, 0.2)
)
df2 <- df[\hlfunctioncall{c}(1,3),]

\hlcomment{# Define the top and bottom of the errorbars}
limits <- \hlfunctioncall{aes}(ymax = resp + se, ymin=resp - se)

p <- \hlfunctioncall{ggplot}(df, \hlfunctioncall{aes}(fill=group, y=resp, x=trt))
p + \hlfunctioncall{geom_bar}(position=\hlstring{"dodge"}, stat=\hlstring{"identity"})
\end{alltt}
\end{kframe}
\includegraphics[width=\maxwidth]{figure/021-ggplot2-geoms-geom_errorbar1} 
\begin{kframe}\begin{alltt}

\hlcomment{# Because the bars and errorbars have different widths}
\hlcomment{# we need to specify how wide the objects we are dodging are}
dodge <- \hlfunctioncall{position_dodge}(width=0.9)
p + \hlfunctioncall{geom_bar}(position=dodge) + \hlfunctioncall{geom_errorbar}(limits, position=dodge, width=0.25)
\end{alltt}
\end{kframe}
\includegraphics[width=\maxwidth]{figure/021-ggplot2-geoms-geom_errorbar2} 
\begin{kframe}\begin{alltt}

p <- \hlfunctioncall{ggplot}(df2, \hlfunctioncall{aes}(fill=group, y=resp, x=trt))
p + \hlfunctioncall{geom_bar}(position=dodge)
\end{alltt}
\end{kframe}
\includegraphics[width=\maxwidth]{figure/021-ggplot2-geoms-geom_errorbar3} 
\begin{kframe}\begin{alltt}
p + \hlfunctioncall{geom_bar}(position=dodge) + \hlfunctioncall{geom_errorbar}(limits, position=dodge, width=0.25)
\end{alltt}
\end{kframe}
\includegraphics[width=\maxwidth]{figure/021-ggplot2-geoms-geom_errorbar4} 
\begin{kframe}\begin{alltt}

p <- \hlfunctioncall{ggplot}(df, \hlfunctioncall{aes}(colour=group, y=resp, x=trt))
p + \hlfunctioncall{geom_point}() + \hlfunctioncall{geom_errorbar}(limits, width=0.2)
\end{alltt}
\end{kframe}
\includegraphics[width=\maxwidth]{figure/021-ggplot2-geoms-geom_errorbar5} 
\begin{kframe}\begin{alltt}
p + \hlfunctioncall{geom_pointrange}(limits)
\end{alltt}
\end{kframe}
\includegraphics[width=\maxwidth]{figure/021-ggplot2-geoms-geom_errorbar6} 
\begin{kframe}\begin{alltt}
p + \hlfunctioncall{geom_crossbar}(limits, width=0.2)
\end{alltt}
\end{kframe}
\includegraphics[width=\maxwidth]{figure/021-ggplot2-geoms-geom_errorbar7} 
\begin{kframe}\begin{alltt}

\hlcomment{# If we want to draw lines, we need to manually set the}
\hlcomment{# groups which define the lines - here the groups in the}
\hlcomment{# original dataframe}
p + \hlfunctioncall{geom_line}(\hlfunctioncall{aes}(group=group)) + \hlfunctioncall{geom_errorbar}(limits, width=0.2)
\end{alltt}
\end{kframe}
\includegraphics[width=\maxwidth]{figure/021-ggplot2-geoms-geom_errorbar8} 
\begin{kframe}\begin{alltt}


\end{alltt}
\end{kframe}
\end{knitrout}



\section{geom\_errorbarh}

\begin{knitrout}
\definecolor{shadecolor}{rgb}{0.969, 0.969, 0.969}\color{fgcolor}\begin{kframe}
\begin{alltt}
\hlcomment{### Name: geom_errorbarh}
\hlcomment{### Title: Horizontal error bars}
\hlcomment{### Aliases: geom_errorbarh}

\hlcomment{### ** Examples}

df <- \hlfunctioncall{data.frame}(
  trt = \hlfunctioncall{factor}(\hlfunctioncall{c}(1, 1, 2, 2)),
  resp = \hlfunctioncall{c}(1, 5, 3, 4),
  group = \hlfunctioncall{factor}(\hlfunctioncall{c}(1, 2, 1, 2)),
  se = \hlfunctioncall{c}(0.1, 0.3, 0.3, 0.2)
)

\hlcomment{# Define the top and bottom of the errorbars}

p <- \hlfunctioncall{ggplot}(df, \hlfunctioncall{aes}(resp, trt, colour = group))
p + \hlfunctioncall{geom_point}() +
  \hlfunctioncall{geom_errorbarh}(\hlfunctioncall{aes}(xmax = resp + se, xmin = resp - se))
\end{alltt}
\end{kframe}
\includegraphics[width=\maxwidth]{figure/021-ggplot2-geoms-geom_errorbarh1} 
\begin{kframe}\begin{alltt}
p + \hlfunctioncall{geom_point}() +
  \hlfunctioncall{geom_errorbarh}(\hlfunctioncall{aes}(xmax = resp + se, xmin = resp - se, height = .2))
\end{alltt}
\end{kframe}
\includegraphics[width=\maxwidth]{figure/021-ggplot2-geoms-geom_errorbarh2} 
\begin{kframe}\begin{alltt}


\end{alltt}
\end{kframe}
\end{knitrout}



\section{geom\_freqpoly}

\begin{knitrout}
\definecolor{shadecolor}{rgb}{0.969, 0.969, 0.969}\color{fgcolor}\begin{kframe}
\begin{alltt}
\hlcomment{### Name: geom_freqpoly}
\hlcomment{### Title: Frequency polygon.}
\hlcomment{### Aliases: geom_freqpoly}

\hlcomment{### ** Examples}

\hlfunctioncall{qplot}(carat, data = diamonds, geom = \hlstring{"freqpoly"})
\end{alltt}


{\ttfamily\noindent\itshape\textcolor{messagecolor}{\#\# stat\_bin: binwidth defaulted to range/30. Use 'binwidth = x' to adjust this.}}\end{kframe}
\includegraphics[width=\maxwidth]{figure/021-ggplot2-geoms-geom_freqpoly1} 
\begin{kframe}\begin{alltt}
\hlfunctioncall{qplot}(carat, data = diamonds, geom = \hlstring{"freqpoly"}, binwidth = 0.1)
\end{alltt}
\end{kframe}
\includegraphics[width=\maxwidth]{figure/021-ggplot2-geoms-geom_freqpoly2} 
\begin{kframe}\begin{alltt}
\hlfunctioncall{qplot}(carat, data = diamonds, geom = \hlstring{"freqpoly"}, binwidth = 0.01)
\end{alltt}
\end{kframe}
\includegraphics[width=\maxwidth]{figure/021-ggplot2-geoms-geom_freqpoly3} 
\begin{kframe}\begin{alltt}

\hlfunctioncall{qplot}(price, data = diamonds, geom = \hlstring{"freqpoly"}, binwidth = 1000)
\end{alltt}
\end{kframe}
\includegraphics[width=\maxwidth]{figure/021-ggplot2-geoms-geom_freqpoly4} 
\begin{kframe}\begin{alltt}
\hlfunctioncall{qplot}(price, data = diamonds, geom = \hlstring{"freqpoly"}, binwidth = 1000,
  colour = color)
\end{alltt}
\end{kframe}
\includegraphics[width=\maxwidth]{figure/021-ggplot2-geoms-geom_freqpoly5} 
\begin{kframe}\begin{alltt}
\hlfunctioncall{qplot}(price, ..density.., data = diamonds, geom = \hlstring{"freqpoly"},
  binwidth = 1000, colour = color)
\end{alltt}
\end{kframe}
\includegraphics[width=\maxwidth]{figure/021-ggplot2-geoms-geom_freqpoly6} 
\begin{kframe}\begin{alltt}


\end{alltt}
\end{kframe}
\end{knitrout}



\section{geom\_hex}

\begin{knitrout}
\definecolor{shadecolor}{rgb}{0.969, 0.969, 0.969}\color{fgcolor}\begin{kframe}
\begin{alltt}
\hlcomment{### Name: geom_hex}
\hlcomment{### Title: Hexagon bining.}
\hlcomment{### Aliases: geom_hex}

\hlcomment{### ** Examples}

\hlcomment{# See ?stat_binhex for examples}



\end{alltt}
\end{kframe}
\end{knitrout}



\section{geom\_histogram}

\begin{knitrout}
\definecolor{shadecolor}{rgb}{0.969, 0.969, 0.969}\color{fgcolor}\begin{kframe}
\begin{alltt}
\hlcomment{### Name: geom_histogram}
\hlcomment{### Title: Histogram}
\hlcomment{### Aliases: geom_histogram}

\hlcomment{### ** Examples}

\hlcomment{## No test: }
\hlfunctioncall{set.seed}(5689)
movies <- movies[\hlfunctioncall{sample}(\hlfunctioncall{nrow}(movies), 1000), ]
\hlcomment{# Simple examples}
\hlfunctioncall{qplot}(rating, data=movies, geom=\hlstring{"histogram"})
\end{alltt}


{\ttfamily\noindent\itshape\textcolor{messagecolor}{\#\# stat\_bin: binwidth defaulted to range/30. Use 'binwidth = x' to adjust this.}}

{\ttfamily\noindent\textcolor{warningcolor}{\#\# Warning: position\_stack requires constant width: output may be incorrect}}\end{kframe}
\includegraphics[width=\maxwidth]{figure/021-ggplot2-geoms-geom_histogram1} 
\begin{kframe}\begin{alltt}
\hlfunctioncall{qplot}(rating, data=movies, weight=votes, geom=\hlstring{"histogram"})
\end{alltt}


{\ttfamily\noindent\itshape\textcolor{messagecolor}{\#\# stat\_bin: binwidth defaulted to range/30. Use 'binwidth = x' to adjust this.}}

{\ttfamily\noindent\textcolor{warningcolor}{\#\# Warning: position\_stack requires constant width: output may be incorrect}}\end{kframe}
\includegraphics[width=\maxwidth]{figure/021-ggplot2-geoms-geom_histogram2} 
\begin{kframe}\begin{alltt}
\hlfunctioncall{qplot}(rating, data=movies, weight=votes, geom=\hlstring{"histogram"}, binwidth=1)
\end{alltt}
\end{kframe}
\includegraphics[width=\maxwidth]{figure/021-ggplot2-geoms-geom_histogram3} 
\begin{kframe}\begin{alltt}
\hlfunctioncall{qplot}(rating, data=movies, weight=votes, geom=\hlstring{"histogram"}, binwidth=0.1)
\end{alltt}


{\ttfamily\noindent\textcolor{warningcolor}{\#\# Warning: position\_stack requires constant width: output may be incorrect}}\end{kframe}
\includegraphics[width=\maxwidth]{figure/021-ggplot2-geoms-geom_histogram4} 
\begin{kframe}\begin{alltt}

\hlcomment{# More complex}
m <- \hlfunctioncall{ggplot}(movies, \hlfunctioncall{aes}(x=rating))
m + \hlfunctioncall{geom_histogram}()
\end{alltt}


{\ttfamily\noindent\itshape\textcolor{messagecolor}{\#\# stat\_bin: binwidth defaulted to range/30. Use 'binwidth = x' to adjust this.}}

{\ttfamily\noindent\textcolor{warningcolor}{\#\# Warning: position\_stack requires constant width: output may be incorrect}}\end{kframe}
\includegraphics[width=\maxwidth]{figure/021-ggplot2-geoms-geom_histogram5} 
\begin{kframe}\begin{alltt}
m + \hlfunctioncall{geom_histogram}(\hlfunctioncall{aes}(y = ..density..)) + \hlfunctioncall{geom_density}()
\end{alltt}


{\ttfamily\noindent\itshape\textcolor{messagecolor}{\#\# stat\_bin: binwidth defaulted to range/30. Use 'binwidth = x' to adjust this.}}

{\ttfamily\noindent\textcolor{warningcolor}{\#\# Warning: position\_stack requires constant width: output may be incorrect}}\end{kframe}
\includegraphics[width=\maxwidth]{figure/021-ggplot2-geoms-geom_histogram6} 
\begin{kframe}\begin{alltt}

m + \hlfunctioncall{geom_histogram}(binwidth = 1)
\end{alltt}
\end{kframe}
\includegraphics[width=\maxwidth]{figure/021-ggplot2-geoms-geom_histogram7} 
\begin{kframe}\begin{alltt}
m + \hlfunctioncall{geom_histogram}(binwidth = 0.5)
\end{alltt}
\end{kframe}
\includegraphics[width=\maxwidth]{figure/021-ggplot2-geoms-geom_histogram8} 
\begin{kframe}\begin{alltt}
m + \hlfunctioncall{geom_histogram}(binwidth = 0.1)
\end{alltt}


{\ttfamily\noindent\textcolor{warningcolor}{\#\# Warning: position\_stack requires constant width: output may be incorrect}}\end{kframe}
\includegraphics[width=\maxwidth]{figure/021-ggplot2-geoms-geom_histogram9} 
\begin{kframe}\begin{alltt}

\hlcomment{# Add aesthetic mappings}
m + \hlfunctioncall{geom_histogram}(\hlfunctioncall{aes}(weight = votes))
\end{alltt}


{\ttfamily\noindent\itshape\textcolor{messagecolor}{\#\# stat\_bin: binwidth defaulted to range/30. Use 'binwidth = x' to adjust this.}}

{\ttfamily\noindent\textcolor{warningcolor}{\#\# Warning: position\_stack requires constant width: output may be incorrect}}\end{kframe}
\includegraphics[width=\maxwidth]{figure/021-ggplot2-geoms-geom_histogram10} 
\begin{kframe}\begin{alltt}
m + \hlfunctioncall{geom_histogram}(\hlfunctioncall{aes}(y = ..count..))
\end{alltt}


{\ttfamily\noindent\itshape\textcolor{messagecolor}{\#\# stat\_bin: binwidth defaulted to range/30. Use 'binwidth = x' to adjust this.}}

{\ttfamily\noindent\textcolor{warningcolor}{\#\# Warning: position\_stack requires constant width: output may be incorrect}}\end{kframe}
\includegraphics[width=\maxwidth]{figure/021-ggplot2-geoms-geom_histogram11} 
\begin{kframe}\begin{alltt}
m + \hlfunctioncall{geom_histogram}(\hlfunctioncall{aes}(fill = ..count..))
\end{alltt}


{\ttfamily\noindent\itshape\textcolor{messagecolor}{\#\# stat\_bin: binwidth defaulted to range/30. Use 'binwidth = x' to adjust this.}}

{\ttfamily\noindent\textcolor{warningcolor}{\#\# Warning: position\_stack requires constant width: output may be incorrect}}\end{kframe}
\includegraphics[width=\maxwidth]{figure/021-ggplot2-geoms-geom_histogram12} 
\begin{kframe}\begin{alltt}

\hlcomment{# Change scales}
m + \hlfunctioncall{geom_histogram}(\hlfunctioncall{aes}(fill = ..count..)) +
  \hlfunctioncall{scale_fill_gradient}(\hlstring{"Count"}, low = \hlstring{"green"}, high = \hlstring{"red"})
\end{alltt}


{\ttfamily\noindent\itshape\textcolor{messagecolor}{\#\# stat\_bin: binwidth defaulted to range/30. Use 'binwidth = x' to adjust this.}}

{\ttfamily\noindent\textcolor{warningcolor}{\#\# Warning: position\_stack requires constant width: output may be incorrect}}\end{kframe}
\includegraphics[width=\maxwidth]{figure/021-ggplot2-geoms-geom_histogram13} 
\begin{kframe}\begin{alltt}

\hlcomment{# Often we don't want the height of the bar to represent the}
\hlcomment{# count of observations, but the sum of some other variable.}
\hlcomment{# For example, the following plot shows the number of movies}
\hlcomment{# in each rating.}
\hlfunctioncall{qplot}(rating, data=movies, geom=\hlstring{"bar"}, binwidth = 0.1)
\end{alltt}


{\ttfamily\noindent\textcolor{warningcolor}{\#\# Warning: position\_stack requires constant width: output may be incorrect}}\end{kframe}
\includegraphics[width=\maxwidth]{figure/021-ggplot2-geoms-geom_histogram14} 
\begin{kframe}\begin{alltt}
\hlcomment{# If, however, we want to see the number of votes cast in each}
\hlcomment{# category, we need to weight by the votes variable}
\hlfunctioncall{qplot}(rating, data=movies, geom=\hlstring{"bar"}, binwidth = 0.1,
  weight=votes, ylab = \hlstring{"votes"})
\end{alltt}


{\ttfamily\noindent\textcolor{warningcolor}{\#\# Warning: position\_stack requires constant width: output may be incorrect}}\end{kframe}
\includegraphics[width=\maxwidth]{figure/021-ggplot2-geoms-geom_histogram15} 
\begin{kframe}\begin{alltt}

m <- \hlfunctioncall{ggplot}(movies, \hlfunctioncall{aes}(x = votes))
\hlcomment{# For transformed scales, binwidth applies to the transformed data.}
\hlcomment{# The bins have constant width on the transformed scale.}
m + \hlfunctioncall{geom_histogram}() + \hlfunctioncall{scale_x_log10}()
\end{alltt}


{\ttfamily\noindent\itshape\textcolor{messagecolor}{\#\# stat\_bin: binwidth defaulted to range/30. Use 'binwidth = x' to adjust this.}}\end{kframe}
\includegraphics[width=\maxwidth]{figure/021-ggplot2-geoms-geom_histogram16} 
\begin{kframe}\begin{alltt}
m + \hlfunctioncall{geom_histogram}(binwidth = 1) + \hlfunctioncall{scale_x_log10}()
\end{alltt}
\end{kframe}
\includegraphics[width=\maxwidth]{figure/021-ggplot2-geoms-geom_histogram17} 
\begin{kframe}\begin{alltt}
m + \hlfunctioncall{geom_histogram}() + \hlfunctioncall{scale_x_sqrt}()
\end{alltt}


{\ttfamily\noindent\itshape\textcolor{messagecolor}{\#\# stat\_bin: binwidth defaulted to range/30. Use 'binwidth = x' to adjust this.}}\end{kframe}
\includegraphics[width=\maxwidth]{figure/021-ggplot2-geoms-geom_histogram18} 
\begin{kframe}\begin{alltt}
m + \hlfunctioncall{geom_histogram}(binwidth = 10) + \hlfunctioncall{scale_x_sqrt}()
\end{alltt}
\end{kframe}
\includegraphics[width=\maxwidth]{figure/021-ggplot2-geoms-geom_histogram19} 
\begin{kframe}\begin{alltt}

\hlcomment{# For transformed coordinate systems, the binwidth applies to the}
\hlcomment{# raw data.  The bins have constant width on the original scale.}

\hlcomment{# Using log scales does not work here, because the first}
\hlcomment{# bar is anchored at zero, and so when transformed becomes negative}
\hlcomment{# infinity.  This is not a problem when transforming the scales, because}
\hlcomment{# no observations have 0 ratings.}
m + \hlfunctioncall{geom_histogram}(origin = 0) + \hlfunctioncall{coord_trans}(x = \hlstring{"log10"})
\end{alltt}


{\ttfamily\noindent\itshape\textcolor{messagecolor}{\#\# stat\_bin: binwidth defaulted to range/30. Use 'binwidth = x' to adjust this.}}\end{kframe}
\includegraphics[width=\maxwidth]{figure/021-ggplot2-geoms-geom_histogram20} 
\begin{kframe}\begin{alltt}
\hlcomment{# Use origin = 0, to make sure we don't take sqrt of negative values}
m + \hlfunctioncall{geom_histogram}(origin = 0) + \hlfunctioncall{coord_trans}(x = \hlstring{"sqrt"})
\end{alltt}


{\ttfamily\noindent\itshape\textcolor{messagecolor}{\#\# stat\_bin: binwidth defaulted to range/30. Use 'binwidth = x' to adjust this.}}\end{kframe}
\includegraphics[width=\maxwidth]{figure/021-ggplot2-geoms-geom_histogram21} 
\begin{kframe}\begin{alltt}
m + \hlfunctioncall{geom_histogram}(origin = 0, binwidth = 1000) + \hlfunctioncall{coord_trans}(x = \hlstring{"sqrt"})
\end{alltt}
\end{kframe}
\includegraphics[width=\maxwidth]{figure/021-ggplot2-geoms-geom_histogram22} 
\begin{kframe}\begin{alltt}

\hlcomment{# You can also transform the y axis.  Remember that the base of the bars}
\hlcomment{# has value 0, so log transformations are not appropriate}
m <- \hlfunctioncall{ggplot}(movies, \hlfunctioncall{aes}(x = rating))
m + \hlfunctioncall{geom_histogram}(binwidth = 0.5) + \hlfunctioncall{scale_y_sqrt}()
\end{alltt}
\end{kframe}
\includegraphics[width=\maxwidth]{figure/021-ggplot2-geoms-geom_histogram23} 
\begin{kframe}\begin{alltt}
m + \hlfunctioncall{geom_histogram}(binwidth = 0.5) + \hlfunctioncall{scale_y_reverse}()
\end{alltt}


{\ttfamily\noindent\textcolor{warningcolor}{\#\# Warning: Stacking not well defined when ymin != 0}}\end{kframe}
\includegraphics[width=\maxwidth]{figure/021-ggplot2-geoms-geom_histogram24} 
\begin{kframe}\begin{alltt}

\hlcomment{# Set aesthetics to fixed value}
m + \hlfunctioncall{geom_histogram}(colour = \hlstring{"darkgreen"}, fill = \hlstring{"white"}, binwidth = 0.5)
\end{alltt}
\end{kframe}
\includegraphics[width=\maxwidth]{figure/021-ggplot2-geoms-geom_histogram25} 
\begin{kframe}\begin{alltt}

\hlcomment{# Use facets}
m <- m + \hlfunctioncall{geom_histogram}(binwidth = 0.5)
m + \hlfunctioncall{facet_grid}(Action ~ Comedy)
\end{alltt}
\end{kframe}
\includegraphics[width=\maxwidth]{figure/021-ggplot2-geoms-geom_histogram26} 
\begin{kframe}\begin{alltt}

\hlcomment{# Often more useful to use density on the y axis when facetting}
m <- m + \hlfunctioncall{aes}(y = ..density..)
m + \hlfunctioncall{facet_grid}(Action ~ Comedy)
\end{alltt}
\end{kframe}
\includegraphics[width=\maxwidth]{figure/021-ggplot2-geoms-geom_histogram27} 
\begin{kframe}\begin{alltt}
m + \hlfunctioncall{facet_wrap}(~ mpaa)
\end{alltt}
\end{kframe}
\includegraphics[width=\maxwidth]{figure/021-ggplot2-geoms-geom_histogram28} 
\begin{kframe}\begin{alltt}

\hlcomment{# Multiple histograms on the same graph}
\hlcomment{# see ?position, ?position_fill, etc for more details.}
\hlfunctioncall{set.seed}(6298)
diamonds_small <- diamonds[\hlfunctioncall{sample}(\hlfunctioncall{nrow}(diamonds), 1000), ]
\hlfunctioncall{ggplot}(diamonds_small, \hlfunctioncall{aes}(x=price)) + \hlfunctioncall{geom_bar}()
\end{alltt}


{\ttfamily\noindent\itshape\textcolor{messagecolor}{\#\# stat\_bin: binwidth defaulted to range/30. Use 'binwidth = x' to adjust this.}}\end{kframe}
\includegraphics[width=\maxwidth]{figure/021-ggplot2-geoms-geom_histogram29} 
\begin{kframe}\begin{alltt}
hist_cut <- \hlfunctioncall{ggplot}(diamonds_small, \hlfunctioncall{aes}(x=price, fill=cut))
hist_cut + \hlfunctioncall{geom_bar}() \hlcomment{# defaults to stacking}
\end{alltt}


{\ttfamily\noindent\itshape\textcolor{messagecolor}{\#\# stat\_bin: binwidth defaulted to range/30. Use 'binwidth = x' to adjust this.}}\end{kframe}
\includegraphics[width=\maxwidth]{figure/021-ggplot2-geoms-geom_histogram30} 
\begin{kframe}\begin{alltt}
hist_cut + \hlfunctioncall{geom_bar}(position=\hlstring{"fill"})
\end{alltt}


{\ttfamily\noindent\itshape\textcolor{messagecolor}{\#\# stat\_bin: binwidth defaulted to range/30. Use 'binwidth = x' to adjust this.}}\end{kframe}
\includegraphics[width=\maxwidth]{figure/021-ggplot2-geoms-geom_histogram31} 
\begin{kframe}\begin{alltt}
hist_cut + \hlfunctioncall{geom_bar}(position=\hlstring{"dodge"})
\end{alltt}


{\ttfamily\noindent\itshape\textcolor{messagecolor}{\#\# stat\_bin: binwidth defaulted to range/30. Use 'binwidth = x' to adjust this.}}\end{kframe}
\includegraphics[width=\maxwidth]{figure/021-ggplot2-geoms-geom_histogram32} 
\begin{kframe}\begin{alltt}

\hlcomment{# This is easy in ggplot2, but not visually effective.  It's better}
\hlcomment{# to use a frequency polygon or density plot.  Like this:}
\hlfunctioncall{ggplot}(diamonds_small, \hlfunctioncall{aes}(price, ..density.., colour = cut)) +
  \hlfunctioncall{geom_freqpoly}(binwidth = 1000)
\end{alltt}
\end{kframe}
\includegraphics[width=\maxwidth]{figure/021-ggplot2-geoms-geom_histogram33} 
\begin{kframe}\begin{alltt}
\hlcomment{# Or this:}
\hlfunctioncall{ggplot}(diamonds_small, \hlfunctioncall{aes}(price, colour = cut)) +
  \hlfunctioncall{geom_density}()
\end{alltt}
\end{kframe}
\includegraphics[width=\maxwidth]{figure/021-ggplot2-geoms-geom_histogram34} 
\begin{kframe}\begin{alltt}
\hlcomment{# Or if you want to be fancy, maybe even this:}
\hlfunctioncall{ggplot}(diamonds_small, \hlfunctioncall{aes}(price, fill = cut)) +
  \hlfunctioncall{geom_density}(alpha = 0.2)
\end{alltt}
\end{kframe}
\includegraphics[width=\maxwidth]{figure/021-ggplot2-geoms-geom_histogram35} 
\begin{kframe}\begin{alltt}
\hlcomment{# Which looks better when the distributions are more distinct}
\hlfunctioncall{ggplot}(diamonds_small, \hlfunctioncall{aes}(depth, fill = cut)) +
  \hlfunctioncall{geom_density}(alpha = 0.2) + \hlfunctioncall{xlim}(55, 70)
\end{alltt}


{\ttfamily\noindent\textcolor{warningcolor}{\#\# Warning: Removed 2 rows containing non-finite values (stat\_density).}}\end{kframe}
\includegraphics[width=\maxwidth]{figure/021-ggplot2-geoms-geom_histogram36} 
\begin{kframe}\begin{alltt}
\hlcomment{## End(No test)}


\end{alltt}
\end{kframe}
\end{knitrout}



\section{geom\_hline}

\begin{knitrout}
\definecolor{shadecolor}{rgb}{0.969, 0.969, 0.969}\color{fgcolor}\begin{kframe}
\begin{alltt}
\hlcomment{### Name: geom_hline}
\hlcomment{### Title: Horizontal line.}
\hlcomment{### Aliases: geom_hline}

\hlcomment{### ** Examples}

p <- \hlfunctioncall{ggplot}(mtcars, \hlfunctioncall{aes}(x = wt, y=mpg)) + \hlfunctioncall{geom_point}()

p + \hlfunctioncall{geom_hline}(\hlfunctioncall{aes}(yintercept=mpg))
\end{alltt}
\end{kframe}
\includegraphics[width=\maxwidth]{figure/021-ggplot2-geoms-geom_hline1} 
\begin{kframe}\begin{alltt}
p + \hlfunctioncall{geom_hline}(yintercept=20)
\end{alltt}
\end{kframe}
\includegraphics[width=\maxwidth]{figure/021-ggplot2-geoms-geom_hline2} 
\begin{kframe}\begin{alltt}
p + \hlfunctioncall{geom_hline}(yintercept=\hlfunctioncall{seq}(10, 30, by=5))
\end{alltt}
\end{kframe}
\includegraphics[width=\maxwidth]{figure/021-ggplot2-geoms-geom_hline3} 
\begin{kframe}\begin{alltt}

\hlcomment{# With coordinate transforms}
p + \hlfunctioncall{geom_hline}(\hlfunctioncall{aes}(yintercept=mpg)) + \hlfunctioncall{coord_equal}()
\end{alltt}
\end{kframe}
\includegraphics[width=\maxwidth]{figure/021-ggplot2-geoms-geom_hline4} 
\begin{kframe}\begin{alltt}
p + \hlfunctioncall{geom_hline}(\hlfunctioncall{aes}(yintercept=mpg)) + \hlfunctioncall{coord_flip}()
\end{alltt}
\end{kframe}
\includegraphics[width=\maxwidth]{figure/021-ggplot2-geoms-geom_hline5} 
\begin{kframe}\begin{alltt}
p + \hlfunctioncall{geom_hline}(\hlfunctioncall{aes}(yintercept=mpg)) + \hlfunctioncall{coord_polar}()
\end{alltt}
\end{kframe}
\includegraphics[width=\maxwidth]{figure/021-ggplot2-geoms-geom_hline6} 
\begin{kframe}\begin{alltt}

\hlcomment{# To display different lines in different facets, you need to}
\hlcomment{# create a data frame.}
p <- \hlfunctioncall{qplot}(mpg, wt, data=mtcars, facets = vs ~ am)

hline.data <- \hlfunctioncall{data.frame}(z = 1:4, vs = \hlfunctioncall{c}(0,0,1,1), am = \hlfunctioncall{c}(0,1,0,1))
p + \hlfunctioncall{geom_hline}(\hlfunctioncall{aes}(yintercept = z), hline.data)
\end{alltt}
\end{kframe}
\includegraphics[width=\maxwidth]{figure/021-ggplot2-geoms-geom_hline7} 
\begin{kframe}\begin{alltt}


\end{alltt}
\end{kframe}
\end{knitrout}



\section{geom\_jitter}

\begin{knitrout}
\definecolor{shadecolor}{rgb}{0.969, 0.969, 0.969}\color{fgcolor}\begin{kframe}
\begin{alltt}
\hlcomment{### Name: geom_jitter}
\hlcomment{### Title: Points, jittered to reduce overplotting.}
\hlcomment{### Aliases: geom_jitter}

\hlcomment{### ** Examples}

p <- \hlfunctioncall{ggplot}(mpg, \hlfunctioncall{aes}(displ, hwy))
p + \hlfunctioncall{geom_point}()
\end{alltt}
\end{kframe}
\includegraphics[width=\maxwidth]{figure/021-ggplot2-geoms-geom_jitter1} 
\begin{kframe}\begin{alltt}
p + \hlfunctioncall{geom_point}(position = \hlstring{"jitter"})
\end{alltt}
\end{kframe}
\includegraphics[width=\maxwidth]{figure/021-ggplot2-geoms-geom_jitter2} 
\begin{kframe}\begin{alltt}

\hlcomment{# Add aesthetic mappings}
p + \hlfunctioncall{geom_jitter}(\hlfunctioncall{aes}(colour = cyl))
\end{alltt}
\end{kframe}
\includegraphics[width=\maxwidth]{figure/021-ggplot2-geoms-geom_jitter3} 
\begin{kframe}\begin{alltt}

\hlcomment{# Vary parameters}
p + \hlfunctioncall{geom_jitter}(position = \hlfunctioncall{position_jitter}(width = .5))
\end{alltt}
\end{kframe}
\includegraphics[width=\maxwidth]{figure/021-ggplot2-geoms-geom_jitter4} 
\begin{kframe}\begin{alltt}
p + \hlfunctioncall{geom_jitter}(position = \hlfunctioncall{position_jitter}(height = .5))
\end{alltt}
\end{kframe}
\includegraphics[width=\maxwidth]{figure/021-ggplot2-geoms-geom_jitter5} 
\begin{kframe}\begin{alltt}

\hlcomment{# Use qplot instead}
\hlfunctioncall{qplot}(displ, hwy, data = mpg, geom = \hlstring{"jitter"})
\end{alltt}
\end{kframe}
\includegraphics[width=\maxwidth]{figure/021-ggplot2-geoms-geom_jitter6} 
\begin{kframe}\begin{alltt}
\hlfunctioncall{qplot}(class, hwy, data = mpg, geom = \hlstring{"jitter"})
\end{alltt}
\end{kframe}
\includegraphics[width=\maxwidth]{figure/021-ggplot2-geoms-geom_jitter7} 
\begin{kframe}\begin{alltt}
\hlfunctioncall{qplot}(class, hwy, data = mpg, geom = \hlfunctioncall{c}(\hlstring{"boxplot"}, \hlstring{"jitter"}))
\end{alltt}
\end{kframe}
\includegraphics[width=\maxwidth]{figure/021-ggplot2-geoms-geom_jitter8} 
\begin{kframe}\begin{alltt}
\hlfunctioncall{qplot}(class, hwy, data = mpg, geom = \hlfunctioncall{c}(\hlstring{"jitter"}, \hlstring{"boxplot"}))
\end{alltt}
\end{kframe}
\includegraphics[width=\maxwidth]{figure/021-ggplot2-geoms-geom_jitter9} 
\begin{kframe}\begin{alltt}


\end{alltt}
\end{kframe}
\end{knitrout}



\section{geom\_line}

\begin{knitrout}
\definecolor{shadecolor}{rgb}{0.969, 0.969, 0.969}\color{fgcolor}\begin{kframe}
\begin{alltt}
\hlcomment{### Name: geom_line}
\hlcomment{### Title: Connect observations, ordered by x value.}
\hlcomment{### Aliases: geom_line}

\hlcomment{### ** Examples}

\hlcomment{# Summarise number of movie ratings by year of movie}
mry <- \hlfunctioncall{do.call}(rbind, \hlfunctioncall{by}(movies, \hlfunctioncall{round}(movies$rating), \hlfunctioncall{function}(df) \{
  nums <- \hlfunctioncall{tapply}(df$length, df$year, length)
  \hlfunctioncall{data.frame}(rating=\hlfunctioncall{round}(df$rating[1]), year = \hlfunctioncall{as.numeric}(\hlfunctioncall{names}(nums)), number=\hlfunctioncall{as.vector}(nums))
\}))

p <- \hlfunctioncall{ggplot}(mry, \hlfunctioncall{aes}(x=year, y=number, group=rating))
p + \hlfunctioncall{geom_line}()
\end{alltt}
\end{kframe}
\includegraphics[width=\maxwidth]{figure/021-ggplot2-geoms-geom_line1} 
\begin{kframe}\begin{alltt}

\hlcomment{# Add aesthetic mappings}
p + \hlfunctioncall{geom_line}(\hlfunctioncall{aes}(size = rating))
\end{alltt}
\end{kframe}
\includegraphics[width=\maxwidth]{figure/021-ggplot2-geoms-geom_line2} 
\begin{kframe}\begin{alltt}
p + \hlfunctioncall{geom_line}(\hlfunctioncall{aes}(colour = rating))
\end{alltt}
\end{kframe}
\includegraphics[width=\maxwidth]{figure/021-ggplot2-geoms-geom_line3} 
\begin{kframe}\begin{alltt}

\hlcomment{# Change scale}
p + \hlfunctioncall{geom_line}(\hlfunctioncall{aes}(colour = rating)) + \hlfunctioncall{scale_colour_gradient}(low=\hlstring{"red"})
\end{alltt}
\end{kframe}
\includegraphics[width=\maxwidth]{figure/021-ggplot2-geoms-geom_line4} 
\begin{kframe}\begin{alltt}
p + \hlfunctioncall{geom_line}(\hlfunctioncall{aes}(size = rating)) + \hlfunctioncall{scale_size}(range = \hlfunctioncall{c}(0.1, 3))
\end{alltt}
\end{kframe}
\includegraphics[width=\maxwidth]{figure/021-ggplot2-geoms-geom_line5} 
\begin{kframe}\begin{alltt}

\hlcomment{# Set aesthetics to fixed value}
p + \hlfunctioncall{geom_line}(colour = \hlstring{"red"}, size = 1)
\end{alltt}
\end{kframe}
\includegraphics[width=\maxwidth]{figure/021-ggplot2-geoms-geom_line6} 
\begin{kframe}\begin{alltt}

\hlcomment{# Use qplot instead}
\hlfunctioncall{qplot}(year, number, data=mry, group=rating, geom=\hlstring{"line"})
\end{alltt}
\end{kframe}
\includegraphics[width=\maxwidth]{figure/021-ggplot2-geoms-geom_line7} 
\begin{kframe}\begin{alltt}

\hlcomment{# Using a time series}
\hlfunctioncall{qplot}(date, pop, data=economics, geom=\hlstring{"line"})
\end{alltt}
\end{kframe}
\includegraphics[width=\maxwidth]{figure/021-ggplot2-geoms-geom_line8} 
\begin{kframe}\begin{alltt}
\hlfunctioncall{qplot}(date, pop, data=economics, geom=\hlstring{"line"}, log=\hlstring{"y"})
\end{alltt}
\end{kframe}
\includegraphics[width=\maxwidth]{figure/021-ggplot2-geoms-geom_line9} 
\begin{kframe}\begin{alltt}
\hlfunctioncall{qplot}(date, pop, data=\hlfunctioncall{subset}(economics, date > \hlfunctioncall{as.Date}(\hlstring{"2006-1-1"})), geom=\hlstring{"line"})
\end{alltt}
\end{kframe}
\includegraphics[width=\maxwidth]{figure/021-ggplot2-geoms-geom_line10} 
\begin{kframe}\begin{alltt}
\hlfunctioncall{qplot}(date, pop, data=economics, size=unemploy/pop, geom=\hlstring{"line"})
\end{alltt}
\end{kframe}
\includegraphics[width=\maxwidth]{figure/021-ggplot2-geoms-geom_line11} 
\begin{kframe}\begin{alltt}

\hlcomment{# Use the arrow parameter to add an arrow to the line}
\hlcomment{# See ?grid::arrow for more details}
c <- \hlfunctioncall{ggplot}(economics, \hlfunctioncall{aes}(x = date, y = pop))
\hlcomment{# Arrow defaults to "last"}
\hlfunctioncall{library}(grid)
c + \hlfunctioncall{geom_line}(arrow = \hlfunctioncall{arrow}())
\end{alltt}
\end{kframe}
\includegraphics[width=\maxwidth]{figure/021-ggplot2-geoms-geom_line12} 
\begin{kframe}\begin{alltt}
c + \hlfunctioncall{geom_line}(arrow = \hlfunctioncall{arrow}(angle = 15, ends = \hlstring{"both"}, type = \hlstring{"closed"}))
\end{alltt}
\end{kframe}
\includegraphics[width=\maxwidth]{figure/021-ggplot2-geoms-geom_line13} 
\begin{kframe}\begin{alltt}

\hlcomment{# See scale_date for examples of plotting multiple times series on}
\hlcomment{# a single graph}

\hlcomment{# A simple pcp example}

y2005 <- \hlfunctioncall{runif}(300, 20, 120)
y2010 <- y2005 * \hlfunctioncall{runif}(300, -1.05, 1.5)
group <- \hlfunctioncall{rep}(LETTERS[1:3], each = 100)

df <- \hlfunctioncall{data.frame}(id = \hlfunctioncall{seq_along}(group), group, y2005, y2010)
\hlfunctioncall{library}(reshape2) \hlcomment{# for melt}
dfm <- \hlfunctioncall{melt}(df, id.var = \hlfunctioncall{c}(\hlstring{"id"}, \hlstring{"group"}))
\hlfunctioncall{ggplot}(dfm, \hlfunctioncall{aes}(variable, value, group = id, colour = group)) +
  \hlfunctioncall{geom_path}(alpha = 0.5)
\end{alltt}
\end{kframe}
\includegraphics[width=\maxwidth]{figure/021-ggplot2-geoms-geom_line14} 
\begin{kframe}\begin{alltt}


\end{alltt}
\end{kframe}
\end{knitrout}



\section{geom\_linerange}

\begin{knitrout}
\definecolor{shadecolor}{rgb}{0.969, 0.969, 0.969}\color{fgcolor}\begin{kframe}
\begin{alltt}
\hlcomment{### Name: geom_linerange}
\hlcomment{### Title: An interval represented by a vertical line.}
\hlcomment{### Aliases: geom_linerange}

\hlcomment{### ** Examples}

\hlcomment{# Generate data: means and standard errors of means for prices}
\hlcomment{# for each type of cut}
dmod <- \hlfunctioncall{lm}(price ~ cut, data=diamonds)
cuts <- \hlfunctioncall{data.frame}(cut=\hlfunctioncall{unique}(diamonds$cut), \hlfunctioncall{predict}(dmod, \hlfunctioncall{data.frame}(cut = \hlfunctioncall{unique}(diamonds$cut)), se=TRUE)[\hlfunctioncall{c}(\hlstring{"fit"},\hlstring{"se.fit"})])

\hlfunctioncall{qplot}(cut, fit, data=cuts)
\end{alltt}
\end{kframe}
\includegraphics[width=\maxwidth]{figure/021-ggplot2-geoms-geom_linerange1} 
\begin{kframe}\begin{alltt}
\hlcomment{# With a bar chart, we are comparing lengths, so the y-axis is}
\hlcomment{# automatically extended to include 0}
\hlfunctioncall{qplot}(cut, fit, data=cuts, geom=\hlstring{"bar"})
\end{alltt}
\end{kframe}
\includegraphics[width=\maxwidth]{figure/021-ggplot2-geoms-geom_linerange2} 
\begin{kframe}\begin{alltt}

\hlcomment{# Display estimates and standard errors in various ways}
se <- \hlfunctioncall{ggplot}(cuts, \hlfunctioncall{aes}(cut, fit,
  ymin = fit - se.fit, ymax=fit + se.fit, colour = cut))
se + \hlfunctioncall{geom_linerange}()
\end{alltt}
\end{kframe}
\includegraphics[width=\maxwidth]{figure/021-ggplot2-geoms-geom_linerange3} 
\begin{kframe}\begin{alltt}
se + \hlfunctioncall{geom_pointrange}()
\end{alltt}
\end{kframe}
\includegraphics[width=\maxwidth]{figure/021-ggplot2-geoms-geom_linerange4} 
\begin{kframe}\begin{alltt}
se + \hlfunctioncall{geom_errorbar}(width = 0.5)
\end{alltt}
\end{kframe}
\includegraphics[width=\maxwidth]{figure/021-ggplot2-geoms-geom_linerange5} 
\begin{kframe}\begin{alltt}
se + \hlfunctioncall{geom_crossbar}(width = 0.5)
\end{alltt}
\end{kframe}
\includegraphics[width=\maxwidth]{figure/021-ggplot2-geoms-geom_linerange6} 
\begin{kframe}\begin{alltt}

\hlcomment{# Use coord_flip to flip the x and y axes}
se + \hlfunctioncall{geom_linerange}() + \hlfunctioncall{coord_flip}()
\end{alltt}
\end{kframe}
\includegraphics[width=\maxwidth]{figure/021-ggplot2-geoms-geom_linerange7} 
\begin{kframe}\begin{alltt}


\end{alltt}
\end{kframe}
\end{knitrout}



\section{geom\_map}

\begin{knitrout}
\definecolor{shadecolor}{rgb}{0.969, 0.969, 0.969}\color{fgcolor}\begin{kframe}
\begin{alltt}
\hlcomment{### Name: geom_map}
\hlcomment{### Title: Polygons from a reference map.}
\hlcomment{### Aliases: geom_map}

\hlcomment{### ** Examples}

\hlcomment{# When using geom_polygon, you will typically need two data frames:}
\hlcomment{# one contains the coordinates of each polygon (positions),  and the}
\hlcomment{# other the values associated with each polygon (values).  An id}
\hlcomment{# variable links the two together}

ids <- \hlfunctioncall{factor}(\hlfunctioncall{c}(\hlstring{"1.1"}, \hlstring{"2.1"}, \hlstring{"1.2"}, \hlstring{"2.2"}, \hlstring{"1.3"}, \hlstring{"2.3"}))

values <- \hlfunctioncall{data.frame}(
  id = ids,
  value = \hlfunctioncall{c}(3, 3.1, 3.1, 3.2, 3.15, 3.5)
)

positions <- \hlfunctioncall{data.frame}(
  id = \hlfunctioncall{rep}(ids, each = 4),
  x = \hlfunctioncall{c}(2, 1, 1.1, 2.2, 1, 0, 0.3, 1.1, 2.2, 1.1, 1.2, 2.5, 1.1, 0.3,
  0.5, 1.2, 2.5, 1.2, 1.3, 2.7, 1.2, 0.5, 0.6, 1.3),
  y = \hlfunctioncall{c}(-0.5, 0, 1, 0.5, 0, 0.5, 1.5, 1, 0.5, 1, 2.1, 1.7, 1, 1.5,
  2.2, 2.1, 1.7, 2.1, 3.2, 2.8, 2.1, 2.2, 3.3, 3.2)
)

\hlfunctioncall{ggplot}(values) + \hlfunctioncall{geom_map}(\hlfunctioncall{aes}(map_id = id), map = positions) +
  \hlfunctioncall{expand_limits}(positions)
\end{alltt}
\end{kframe}
\includegraphics[width=\maxwidth]{figure/021-ggplot2-geoms-geom_map1} 
\begin{kframe}\begin{alltt}
\hlfunctioncall{ggplot}(values, \hlfunctioncall{aes}(fill = value)) +
  \hlfunctioncall{geom_map}(\hlfunctioncall{aes}(map_id = id), map = positions) +
  \hlfunctioncall{expand_limits}(positions)
\end{alltt}
\end{kframe}
\includegraphics[width=\maxwidth]{figure/021-ggplot2-geoms-geom_map2} 
\begin{kframe}\begin{alltt}
\hlfunctioncall{ggplot}(values, \hlfunctioncall{aes}(fill = value)) +
  \hlfunctioncall{geom_map}(\hlfunctioncall{aes}(map_id = id), map = positions) +
  \hlfunctioncall{expand_limits}(positions) + \hlfunctioncall{ylim}(0, 3)
\end{alltt}
\end{kframe}
\includegraphics[width=\maxwidth]{figure/021-ggplot2-geoms-geom_map3} 
\begin{kframe}\begin{alltt}

\hlcomment{# Better example}
crimes <- \hlfunctioncall{data.frame}(state = \hlfunctioncall{tolower}(\hlfunctioncall{rownames}(USArrests)), USArrests)
\hlfunctioncall{library}(reshape2) \hlcomment{# for melt}
crimesm <- \hlfunctioncall{melt}(crimes, id = 1)
\hlfunctioncall{if} (\hlfunctioncall{require}(maps)) \{
  states_map <- \hlfunctioncall{map_data}(\hlstring{"state"})
  \hlfunctioncall{ggplot}(crimes, \hlfunctioncall{aes}(map_id = state)) + \hlfunctioncall{geom_map}(\hlfunctioncall{aes}(fill = Murder), map = states_map) + \hlfunctioncall{expand_limits}(x = states_map$long, y = states_map$lat)
  \hlfunctioncall{last_plot}() + \hlfunctioncall{coord_map}()
  \hlfunctioncall{ggplot}(crimesm, \hlfunctioncall{aes}(map_id = state)) + \hlfunctioncall{geom_map}(\hlfunctioncall{aes}(fill = value), map = states_map) + \hlfunctioncall{expand_limits}(x = states_map$long, y = states_map$lat) + \hlfunctioncall{facet_wrap}( ~ variable)
\}
\end{alltt}


{\ttfamily\noindent\itshape\textcolor{messagecolor}{\#\# Loading required package: maps}}\end{kframe}
\includegraphics[width=\maxwidth]{figure/021-ggplot2-geoms-geom_map4} 
\begin{kframe}\begin{alltt}


\end{alltt}
\end{kframe}
\end{knitrout}



\section{geom\_path}

\begin{knitrout}
\definecolor{shadecolor}{rgb}{0.969, 0.969, 0.969}\color{fgcolor}\begin{kframe}
\begin{alltt}
\hlcomment{### Name: geom_path}
\hlcomment{### Title: Connect observations in original order}
\hlcomment{### Aliases: geom_path}

\hlcomment{### ** Examples}

\hlcomment{## No test: }
\hlcomment{# Generate data}
\hlfunctioncall{library}(plyr)
myear <- \hlfunctioncall{ddply}(movies, \hlfunctioncall{.}(year), \hlfunctioncall{colwise}(mean, \hlfunctioncall{.}(length, rating)))
p <- \hlfunctioncall{ggplot}(myear, \hlfunctioncall{aes}(length, rating))
p + \hlfunctioncall{geom_path}()
\end{alltt}
\end{kframe}
\includegraphics[width=\maxwidth]{figure/021-ggplot2-geoms-geom_path1} 
\begin{kframe}\begin{alltt}

\hlcomment{# Add aesthetic mappings}
p + \hlfunctioncall{geom_path}(\hlfunctioncall{aes}(size = year))
\end{alltt}
\end{kframe}
\includegraphics[width=\maxwidth]{figure/021-ggplot2-geoms-geom_path2} 
\begin{kframe}\begin{alltt}
p + \hlfunctioncall{geom_path}(\hlfunctioncall{aes}(colour = year))
\end{alltt}
\end{kframe}
\includegraphics[width=\maxwidth]{figure/021-ggplot2-geoms-geom_path3} 
\begin{kframe}\begin{alltt}

\hlcomment{# Change scale}
p + \hlfunctioncall{geom_path}(\hlfunctioncall{aes}(size = year)) + \hlfunctioncall{scale_size}(range = \hlfunctioncall{c}(1, 3))
\end{alltt}
\end{kframe}
\includegraphics[width=\maxwidth]{figure/021-ggplot2-geoms-geom_path4} 
\begin{kframe}\begin{alltt}

\hlcomment{# Set aesthetics to fixed value}
p + \hlfunctioncall{geom_path}(colour = \hlstring{"green"})
\end{alltt}
\end{kframe}
\includegraphics[width=\maxwidth]{figure/021-ggplot2-geoms-geom_path5} 
\begin{kframe}\begin{alltt}

\hlcomment{# Control line join parameters}
df <- \hlfunctioncall{data.frame}(x = 1:3, y = \hlfunctioncall{c}(4, 1, 9))
base <- \hlfunctioncall{ggplot}(df, \hlfunctioncall{aes}(x, y))
base + \hlfunctioncall{geom_path}(size = 10)
\end{alltt}
\end{kframe}
\includegraphics[width=\maxwidth]{figure/021-ggplot2-geoms-geom_path6} 
\begin{kframe}\begin{alltt}
base + \hlfunctioncall{geom_path}(size = 10, lineend = \hlstring{"round"})
\end{alltt}
\end{kframe}
\includegraphics[width=\maxwidth]{figure/021-ggplot2-geoms-geom_path7} 
\begin{kframe}\begin{alltt}
base + \hlfunctioncall{geom_path}(size = 10, linejoin = \hlstring{"mitre"}, lineend = \hlstring{"butt"})
\end{alltt}
\end{kframe}
\includegraphics[width=\maxwidth]{figure/021-ggplot2-geoms-geom_path8} 
\begin{kframe}\begin{alltt}

\hlcomment{# Use qplot instead}
\hlfunctioncall{qplot}(length, rating, data=myear, geom=\hlstring{"path"})
\end{alltt}
\end{kframe}
\includegraphics[width=\maxwidth]{figure/021-ggplot2-geoms-geom_path9} 
\begin{kframe}\begin{alltt}

\hlcomment{# Using economic data:}
\hlcomment{# How is unemployment and personal savings rate related?}
\hlfunctioncall{qplot}(unemploy/pop, psavert, data=economics)
\end{alltt}
\end{kframe}
\includegraphics[width=\maxwidth]{figure/021-ggplot2-geoms-geom_path10} 
\begin{kframe}\begin{alltt}
\hlfunctioncall{qplot}(unemploy/pop, psavert, data=economics, geom=\hlstring{"path"})
\end{alltt}
\end{kframe}
\includegraphics[width=\maxwidth]{figure/021-ggplot2-geoms-geom_path11} 
\begin{kframe}\begin{alltt}
\hlfunctioncall{qplot}(unemploy/pop, psavert, data=economics, geom=\hlstring{"path"}, size=\hlfunctioncall{as.numeric}(date))
\end{alltt}
\end{kframe}
\includegraphics[width=\maxwidth]{figure/021-ggplot2-geoms-geom_path12} 
\begin{kframe}\begin{alltt}

\hlcomment{# How is rate of unemployment and length of unemployment?}
\hlfunctioncall{qplot}(unemploy/pop, uempmed, data=economics)
\end{alltt}
\end{kframe}
\includegraphics[width=\maxwidth]{figure/021-ggplot2-geoms-geom_path13} 
\begin{kframe}\begin{alltt}
\hlfunctioncall{qplot}(unemploy/pop, uempmed, data=economics, geom=\hlstring{"path"})
\end{alltt}
\end{kframe}
\includegraphics[width=\maxwidth]{figure/021-ggplot2-geoms-geom_path14} 
\begin{kframe}\begin{alltt}
\hlfunctioncall{qplot}(unemploy/pop, uempmed, data=economics, geom=\hlstring{"path"}) +
  \hlfunctioncall{geom_point}(data=\hlfunctioncall{head}(economics, 1), colour=\hlstring{"red"}) +
  \hlfunctioncall{geom_point}(data=\hlfunctioncall{tail}(economics, 1), colour=\hlstring{"blue"})
\end{alltt}
\end{kframe}
\includegraphics[width=\maxwidth]{figure/021-ggplot2-geoms-geom_path15} 
\begin{kframe}\begin{alltt}
\hlfunctioncall{qplot}(unemploy/pop, uempmed, data=economics, geom=\hlstring{"path"}) +
  \hlfunctioncall{geom_text}(data=\hlfunctioncall{head}(economics, 1), label=\hlstring{"1967"}, colour=\hlstring{"blue"}) +
  \hlfunctioncall{geom_text}(data=\hlfunctioncall{tail}(economics, 1), label=\hlstring{"2007"}, colour=\hlstring{"blue"})
\end{alltt}
\end{kframe}
\includegraphics[width=\maxwidth]{figure/021-ggplot2-geoms-geom_path16} 
\begin{kframe}\begin{alltt}

\hlcomment{# geom_path removes missing values on the ends of a line.}
\hlcomment{# use na.rm = T to suppress the warning message}
df <- \hlfunctioncall{data.frame}(
  x = 1:5,
  y1 = \hlfunctioncall{c}(1, 2, 3, 4, NA),
  y2 = \hlfunctioncall{c}(NA, 2, 3, 4, 5),
  y3 = \hlfunctioncall{c}(1, 2, NA, 4, 5),
  y4 = \hlfunctioncall{c}(1, 2, 3, 4, 5))
\hlfunctioncall{qplot}(x, y1, data = df, geom = \hlfunctioncall{c}(\hlstring{"point"},\hlstring{"line"}))
\end{alltt}


{\ttfamily\noindent\textcolor{warningcolor}{\#\# Warning: Removed 1 rows containing missing values (geom\_point).}}

{\ttfamily\noindent\textcolor{warningcolor}{\#\# Warning: Removed 1 rows containing missing values (geom\_path).}}\end{kframe}
\includegraphics[width=\maxwidth]{figure/021-ggplot2-geoms-geom_path17} 
\begin{kframe}\begin{alltt}
\hlfunctioncall{qplot}(x, y2, data = df, geom = \hlfunctioncall{c}(\hlstring{"point"},\hlstring{"line"}))
\end{alltt}


{\ttfamily\noindent\textcolor{warningcolor}{\#\# Warning: Removed 1 rows containing missing values (geom\_point).}}

{\ttfamily\noindent\textcolor{warningcolor}{\#\# Warning: Removed 1 rows containing missing values (geom\_path).}}\end{kframe}
\includegraphics[width=\maxwidth]{figure/021-ggplot2-geoms-geom_path18} 
\begin{kframe}\begin{alltt}
\hlfunctioncall{qplot}(x, y3, data = df, geom = \hlfunctioncall{c}(\hlstring{"point"},\hlstring{"line"}))
\end{alltt}


{\ttfamily\noindent\textcolor{warningcolor}{\#\# Warning: Removed 1 rows containing missing values (geom\_point).}}\end{kframe}
\includegraphics[width=\maxwidth]{figure/021-ggplot2-geoms-geom_path19} 
\begin{kframe}\begin{alltt}
\hlfunctioncall{qplot}(x, y4, data = df, geom = \hlfunctioncall{c}(\hlstring{"point"},\hlstring{"line"}))
\end{alltt}
\end{kframe}
\includegraphics[width=\maxwidth]{figure/021-ggplot2-geoms-geom_path20} 
\begin{kframe}\begin{alltt}

\hlcomment{# Setting line type vs colour/size}
\hlcomment{# Line type needs to be applied to a line as a whole, so it can}
\hlcomment{# not be used with colour or size that vary across a line}

x <- \hlfunctioncall{seq}(0.01, .99, length=100)
df <- \hlfunctioncall{data.frame}(x = \hlfunctioncall{rep}(x, 2), y = \hlfunctioncall{c}(\hlfunctioncall{qlogis}(x), 2 * \hlfunctioncall{qlogis}(x)), group = \hlfunctioncall{rep}(\hlfunctioncall{c}(\hlstring{"a"},\hlstring{"b"}), each=100))
p <- \hlfunctioncall{ggplot}(df, \hlfunctioncall{aes}(x=x, y=y, group=group))

\hlcomment{# Should work}
p + \hlfunctioncall{geom_line}(linetype = 2)
\end{alltt}
\end{kframe}
\includegraphics[width=\maxwidth]{figure/021-ggplot2-geoms-geom_path21} 
\begin{kframe}\begin{alltt}
p + \hlfunctioncall{geom_line}(\hlfunctioncall{aes}(colour = group), linetype = 2)
\end{alltt}
\end{kframe}
\includegraphics[width=\maxwidth]{figure/021-ggplot2-geoms-geom_path22} 
\begin{kframe}\begin{alltt}
p + \hlfunctioncall{geom_line}(\hlfunctioncall{aes}(colour = x))
\end{alltt}
\end{kframe}
\includegraphics[width=\maxwidth]{figure/021-ggplot2-geoms-geom_path23} 
\begin{kframe}\begin{alltt}

\hlcomment{# Should fail}
\hlfunctioncall{should_stop}(p + \hlfunctioncall{geom_line}(\hlfunctioncall{aes}(colour = x), linetype=2))

\hlcomment{# Use the arrow parameter to add an arrow to the line}
\hlcomment{# See ?grid::arrow for more details}
\hlfunctioncall{library}(grid)
c <- \hlfunctioncall{ggplot}(economics, \hlfunctioncall{aes}(x = date, y = pop))
\hlcomment{# Arrow defaults to "last"}
c + \hlfunctioncall{geom_path}(arrow = \hlfunctioncall{arrow}())
\end{alltt}
\end{kframe}
\includegraphics[width=\maxwidth]{figure/021-ggplot2-geoms-geom_path24} 
\begin{kframe}\begin{alltt}
c + \hlfunctioncall{geom_path}(arrow = \hlfunctioncall{arrow}(angle = 15, ends = \hlstring{"both"}, length = \hlfunctioncall{unit}(0.6, \hlstring{"inches"})))
\end{alltt}
\end{kframe}
\includegraphics[width=\maxwidth]{figure/021-ggplot2-geoms-geom_path25} 
\begin{kframe}\begin{alltt}
\hlcomment{## End(No test)}


\end{alltt}
\end{kframe}
\end{knitrout}



\section{geom\_point}

\begin{knitrout}
\definecolor{shadecolor}{rgb}{0.969, 0.969, 0.969}\color{fgcolor}\begin{kframe}
\begin{alltt}
\hlcomment{### Name: geom_point}
\hlcomment{### Title: Points, as for a scatterplot}
\hlcomment{### Aliases: geom_point}

\hlcomment{### ** Examples}

\hlcomment{## No test: }
p <- \hlfunctioncall{ggplot}(mtcars, \hlfunctioncall{aes}(wt, mpg))
p + \hlfunctioncall{geom_point}()
\end{alltt}
\end{kframe}
\includegraphics[width=\maxwidth]{figure/021-ggplot2-geoms-geom_point1} 
\begin{kframe}\begin{alltt}

\hlcomment{# Add aesthetic mappings}
p + \hlfunctioncall{geom_point}(\hlfunctioncall{aes}(colour = qsec))
\end{alltt}
\end{kframe}
\includegraphics[width=\maxwidth]{figure/021-ggplot2-geoms-geom_point2} 
\begin{kframe}\begin{alltt}
p + \hlfunctioncall{geom_point}(\hlfunctioncall{aes}(alpha = qsec))
\end{alltt}
\end{kframe}
\includegraphics[width=\maxwidth]{figure/021-ggplot2-geoms-geom_point3} 
\begin{kframe}\begin{alltt}
p + \hlfunctioncall{geom_point}(\hlfunctioncall{aes}(colour = \hlfunctioncall{factor}(cyl)))
\end{alltt}
\end{kframe}
\includegraphics[width=\maxwidth]{figure/021-ggplot2-geoms-geom_point4} 
\begin{kframe}\begin{alltt}
p + \hlfunctioncall{geom_point}(\hlfunctioncall{aes}(shape = \hlfunctioncall{factor}(cyl)))
\end{alltt}
\end{kframe}
\includegraphics[width=\maxwidth]{figure/021-ggplot2-geoms-geom_point5} 
\begin{kframe}\begin{alltt}
p + \hlfunctioncall{geom_point}(\hlfunctioncall{aes}(size = qsec))
\end{alltt}
\end{kframe}
\includegraphics[width=\maxwidth]{figure/021-ggplot2-geoms-geom_point6} 
\begin{kframe}\begin{alltt}

\hlcomment{# Change scales}
p + \hlfunctioncall{geom_point}(\hlfunctioncall{aes}(colour = cyl)) + \hlfunctioncall{scale_colour_gradient}(low = \hlstring{"blue"})
\end{alltt}
\end{kframe}
\includegraphics[width=\maxwidth]{figure/021-ggplot2-geoms-geom_point7} 
\begin{kframe}\begin{alltt}
p + \hlfunctioncall{geom_point}(\hlfunctioncall{aes}(size = qsec)) + \hlfunctioncall{scale_area}()
\end{alltt}
\end{kframe}
\includegraphics[width=\maxwidth]{figure/021-ggplot2-geoms-geom_point8} 
\begin{kframe}\begin{alltt}
p + \hlfunctioncall{geom_point}(\hlfunctioncall{aes}(shape = \hlfunctioncall{factor}(cyl))) + \hlfunctioncall{scale_shape}(solid = FALSE)
\end{alltt}
\end{kframe}
\includegraphics[width=\maxwidth]{figure/021-ggplot2-geoms-geom_point9} 
\begin{kframe}\begin{alltt}

\hlcomment{# Set aesthetics to fixed value}
p + \hlfunctioncall{geom_point}(colour = \hlstring{"red"}, size = 3)
\end{alltt}
\end{kframe}
\includegraphics[width=\maxwidth]{figure/021-ggplot2-geoms-geom_point10} 
\begin{kframe}\begin{alltt}
\hlfunctioncall{qplot}(wt, mpg, data = mtcars, colour = \hlfunctioncall{I}(\hlstring{"red"}), size = \hlfunctioncall{I}(3))
\end{alltt}
\end{kframe}
\includegraphics[width=\maxwidth]{figure/021-ggplot2-geoms-geom_point11} 
\begin{kframe}\begin{alltt}

\hlcomment{# Varying alpha is useful for large datasets}
d <- \hlfunctioncall{ggplot}(diamonds, \hlfunctioncall{aes}(carat, price))
d + \hlfunctioncall{geom_point}(alpha = 1/10)
\end{alltt}
\end{kframe}
\includegraphics[width=\maxwidth]{figure/021-ggplot2-geoms-geom_point12} 
\begin{kframe}\begin{alltt}
d + \hlfunctioncall{geom_point}(alpha = 1/20)
\end{alltt}
\end{kframe}
\includegraphics[width=\maxwidth]{figure/021-ggplot2-geoms-geom_point13} 
\begin{kframe}\begin{alltt}
d + \hlfunctioncall{geom_point}(alpha = 1/100)
\end{alltt}
\end{kframe}
\includegraphics[width=\maxwidth]{figure/021-ggplot2-geoms-geom_point14} 
\begin{kframe}\begin{alltt}

\hlcomment{# You can create interesting shapes by layering multiple points of}
\hlcomment{# different sizes}
p <- \hlfunctioncall{ggplot}(mtcars, \hlfunctioncall{aes}(mpg, wt))
p + \hlfunctioncall{geom_point}(colour=\hlstring{"grey50"}, size = 4) + \hlfunctioncall{geom_point}(\hlfunctioncall{aes}(colour = cyl))
\end{alltt}
\end{kframe}
\includegraphics[width=\maxwidth]{figure/021-ggplot2-geoms-geom_point15} 
\begin{kframe}\begin{alltt}
p + \hlfunctioncall{aes}(shape = \hlfunctioncall{factor}(cyl)) +
  \hlfunctioncall{geom_point}(\hlfunctioncall{aes}(colour = \hlfunctioncall{factor}(cyl)), size = 4) +
  \hlfunctioncall{geom_point}(colour=\hlstring{"grey90"}, size = 1.5)
\end{alltt}
\end{kframe}
\includegraphics[width=\maxwidth]{figure/021-ggplot2-geoms-geom_point16} 
\begin{kframe}\begin{alltt}
p + \hlfunctioncall{geom_point}(colour=\hlstring{"black"}, size = 4.5) +
  \hlfunctioncall{geom_point}(colour=\hlstring{"pink"}, size = 4) +
  \hlfunctioncall{geom_point}(\hlfunctioncall{aes}(shape = \hlfunctioncall{factor}(cyl)))
\end{alltt}
\end{kframe}
\includegraphics[width=\maxwidth]{figure/021-ggplot2-geoms-geom_point17} 
\begin{kframe}\begin{alltt}

\hlcomment{# These extra layers don't usually appear in the legend, but we can}
\hlcomment{# force their inclusion}
p + \hlfunctioncall{geom_point}(colour=\hlstring{"black"}, size = 4.5, show_guide = TRUE) +
  \hlfunctioncall{geom_point}(colour=\hlstring{"pink"}, size = 4, show_guide = TRUE) +
  \hlfunctioncall{geom_point}(\hlfunctioncall{aes}(shape = \hlfunctioncall{factor}(cyl)))
\end{alltt}
\end{kframe}
\includegraphics[width=\maxwidth]{figure/021-ggplot2-geoms-geom_point18} 
\begin{kframe}\begin{alltt}

\hlcomment{# Transparent points:}
\hlfunctioncall{qplot}(mpg, wt, data = mtcars, size = \hlfunctioncall{I}(5), alpha = \hlfunctioncall{I}(0.2))
\end{alltt}
\end{kframe}
\includegraphics[width=\maxwidth]{figure/021-ggplot2-geoms-geom_point19} 
\begin{kframe}\begin{alltt}

\hlcomment{# geom_point warns when missing values have been dropped from the data set}
\hlcomment{# and not plotted, you can turn this off by setting na.rm = TRUE}
mtcars2 <- \hlfunctioncall{transform}(mtcars, mpg = \hlfunctioncall{ifelse}(\hlfunctioncall{runif}(32) < 0.2, NA, mpg))
\hlfunctioncall{qplot}(wt, mpg, data = mtcars2)
\end{alltt}


{\ttfamily\noindent\textcolor{warningcolor}{\#\# Warning: Removed 11 rows containing missing values (geom\_point).}}\end{kframe}
\includegraphics[width=\maxwidth]{figure/021-ggplot2-geoms-geom_point20} 
\begin{kframe}\begin{alltt}
\hlfunctioncall{qplot}(wt, mpg, data = mtcars2, na.rm = TRUE)
\end{alltt}
\end{kframe}
\includegraphics[width=\maxwidth]{figure/021-ggplot2-geoms-geom_point21} 
\begin{kframe}\begin{alltt}

\hlcomment{# Use qplot instead}
\hlfunctioncall{qplot}(wt, mpg, data = mtcars)
\end{alltt}
\end{kframe}
\includegraphics[width=\maxwidth]{figure/021-ggplot2-geoms-geom_point22} 
\begin{kframe}\begin{alltt}
\hlfunctioncall{qplot}(wt, mpg, data = mtcars, colour = \hlfunctioncall{factor}(cyl))
\end{alltt}
\end{kframe}
\includegraphics[width=\maxwidth]{figure/021-ggplot2-geoms-geom_point23} 
\begin{kframe}\begin{alltt}
\hlfunctioncall{qplot}(wt, mpg, data = mtcars, colour = \hlfunctioncall{I}(\hlstring{"red"}))
\end{alltt}
\end{kframe}
\includegraphics[width=\maxwidth]{figure/021-ggplot2-geoms-geom_point24} 
\begin{kframe}\begin{alltt}
\hlcomment{## End(No test)}


\end{alltt}
\end{kframe}
\end{knitrout}



\section{geom\_pointrange}

\begin{knitrout}
\definecolor{shadecolor}{rgb}{0.969, 0.969, 0.969}\color{fgcolor}\begin{kframe}
\begin{alltt}
\hlcomment{### Name: geom_pointrange}
\hlcomment{### Title: An interval represented by a vertical line, with a point in the}
\hlcomment{###   middle.}
\hlcomment{### Aliases: geom_pointrange}

\hlcomment{### ** Examples}

\hlcomment{# See geom_linerange for examples}



\end{alltt}
\end{kframe}
\end{knitrout}



\section{geom\_polygon}

\begin{knitrout}
\definecolor{shadecolor}{rgb}{0.969, 0.969, 0.969}\color{fgcolor}\begin{kframe}
\begin{alltt}
\hlcomment{### Name: geom_polygon}
\hlcomment{### Title: Polygon, a filled path.}
\hlcomment{### Aliases: geom_polygon}

\hlcomment{### ** Examples}

\hlcomment{# When using geom_polygon, you will typically need two data frames:}
\hlcomment{# one contains the coordinates of each polygon (positions),  and the}
\hlcomment{# other the values associated with each polygon (values).  An id}
\hlcomment{# variable links the two together}

ids <- \hlfunctioncall{factor}(\hlfunctioncall{c}(\hlstring{"1.1"}, \hlstring{"2.1"}, \hlstring{"1.2"}, \hlstring{"2.2"}, \hlstring{"1.3"}, \hlstring{"2.3"}))

values <- \hlfunctioncall{data.frame}(
  id = ids,
  value = \hlfunctioncall{c}(3, 3.1, 3.1, 3.2, 3.15, 3.5)
)

positions <- \hlfunctioncall{data.frame}(
  id = \hlfunctioncall{rep}(ids, each = 4),
  x = \hlfunctioncall{c}(2, 1, 1.1, 2.2, 1, 0, 0.3, 1.1, 2.2, 1.1, 1.2, 2.5, 1.1, 0.3,
  0.5, 1.2, 2.5, 1.2, 1.3, 2.7, 1.2, 0.5, 0.6, 1.3),
  y = \hlfunctioncall{c}(-0.5, 0, 1, 0.5, 0, 0.5, 1.5, 1, 0.5, 1, 2.1, 1.7, 1, 1.5,
  2.2, 2.1, 1.7, 2.1, 3.2, 2.8, 2.1, 2.2, 3.3, 3.2)
)

\hlcomment{# Currently we need to manually merge the two together}
datapoly <- \hlfunctioncall{merge}(values, positions, by=\hlfunctioncall{c}(\hlstring{"id"}))

(p <- \hlfunctioncall{ggplot}(datapoly, \hlfunctioncall{aes}(x=x, y=y)) + \hlfunctioncall{geom_polygon}(\hlfunctioncall{aes}(fill=value, group=id)))
\end{alltt}
\end{kframe}
\includegraphics[width=\maxwidth]{figure/021-ggplot2-geoms-geom_polygon1} 
\begin{kframe}\begin{alltt}

\hlcomment{# Which seems like a lot of work, but then it's easy to add on}
\hlcomment{# other features in this coordinate system, e.g.:}

stream <- \hlfunctioncall{data.frame}(
  x = \hlfunctioncall{cumsum}(\hlfunctioncall{runif}(50, max = 0.1)),
  y = \hlfunctioncall{cumsum}(\hlfunctioncall{runif}(50,max = 0.1))
)

p + \hlfunctioncall{geom_line}(data = stream, colour=\hlstring{"grey30"}, size = 5)
\end{alltt}
\end{kframe}
\includegraphics[width=\maxwidth]{figure/021-ggplot2-geoms-geom_polygon2} 
\begin{kframe}\begin{alltt}

\hlcomment{# And if the positions are in longitude and latitude, you can use}
\hlcomment{# coord_map to produce different map projections.}


\end{alltt}
\end{kframe}
\end{knitrout}



\section{geom\_quantile}

\begin{knitrout}
\definecolor{shadecolor}{rgb}{0.969, 0.969, 0.969}\color{fgcolor}\begin{kframe}
\begin{alltt}
\hlcomment{### Name: geom_quantile}
\hlcomment{### Title: Add quantile lines from a quantile regression.}
\hlcomment{### Aliases: geom_quantile}

\hlcomment{### ** Examples}

\hlcomment{# See stat_quantile for examples}



\end{alltt}
\end{kframe}
\end{knitrout}



\section{geom\_raster}

\begin{knitrout}
\definecolor{shadecolor}{rgb}{0.969, 0.969, 0.969}\color{fgcolor}\begin{kframe}
\begin{alltt}
\hlcomment{### Name: geom_raster}
\hlcomment{### Title: High-performance rectangular tiling.}
\hlcomment{### Aliases: geom_raster}

\hlcomment{### ** Examples}

\hlcomment{## No test: }
\hlcomment{# Generate data}
pp <- \hlfunctioncall{function} (n,r=4) \{
 x <- \hlfunctioncall{seq}(-r*pi, r*pi, len=n)
 df <- \hlfunctioncall{expand.grid}(x=x, y=x)
 df$r <- \hlfunctioncall{sqrt}(df$x^2 + df$y^2)
 df$z <- \hlfunctioncall{cos}(df$r^2)*\hlfunctioncall{exp}(-df$r/6)
 df
\}
\hlfunctioncall{qplot}(x, y, data = \hlfunctioncall{pp}(20), fill = z, geom = \hlstring{"raster"})
\end{alltt}
\end{kframe}
\includegraphics[width=\maxwidth]{figure/021-ggplot2-geoms-geom_raster1} 
\begin{kframe}\begin{alltt}
\hlcomment{# Interpolation worsens the apperance of this plot, but can help when}
\hlcomment{# rendering images.}
\hlfunctioncall{qplot}(x, y, data = \hlfunctioncall{pp}(20), fill = z, geom = \hlstring{"raster"}, interpolate = TRUE)

\hlcomment{# For the special cases where it is applicable, geom_raster is much}
\hlcomment{# faster than geom_tile:}
pp200 <- \hlfunctioncall{pp}(200)
base <- \hlfunctioncall{ggplot}(pp200, \hlfunctioncall{aes}(x, y, fill = z))
\hlfunctioncall{benchplot}(base + \hlfunctioncall{geom_raster}())
\end{alltt}
\begin{verbatim}
##        step user.self sys.self elapsed
## 1 construct     0.008    0.000   0.010
## 2     build     1.164    0.020   1.189
## 3    render     0.420    0.016   0.438
## 4      draw     0.116    0.000   0.116
## 5     TOTAL     1.708    0.036   1.753
\end{verbatim}
\begin{alltt}
\hlfunctioncall{benchplot}(base + \hlfunctioncall{geom_tile}())
\end{alltt}
\end{kframe}
\includegraphics[width=\maxwidth]{figure/021-ggplot2-geoms-geom_raster2} 
\begin{kframe}\begin{verbatim}
##        step user.self sys.self elapsed
## 1 construct     0.008    0.000   0.009
## 2     build     1.184    0.012   1.199
## 3    render     0.580    0.024   0.609
## 4      draw     0.248    0.000   0.250
## 5     TOTAL     2.020    0.036   2.067
\end{verbatim}
\begin{alltt}

\hlcomment{# justification}
df <- \hlfunctioncall{expand.grid}(x = 0:5, y = 0:5)
df$z <- \hlfunctioncall{runif}(\hlfunctioncall{nrow}(df))
\hlcomment{# default is compatible with geom_tile()}
\hlfunctioncall{ggplot}(df, \hlfunctioncall{aes}(x, y, fill = z)) + \hlfunctioncall{geom_raster}()
\end{alltt}
\end{kframe}
\includegraphics[width=\maxwidth]{figure/021-ggplot2-geoms-geom_raster3} 
\begin{kframe}\begin{alltt}
\hlcomment{# zero padding}
\hlfunctioncall{ggplot}(df, \hlfunctioncall{aes}(x, y, fill = z)) + \hlfunctioncall{geom_raster}(hjust = 0, vjust = 0)
\end{alltt}
\end{kframe}
\includegraphics[width=\maxwidth]{figure/021-ggplot2-geoms-geom_raster4} 
\begin{kframe}\begin{alltt}
\hlcomment{## End(No test)}


\end{alltt}
\end{kframe}
\end{knitrout}



\section{geom\_rect}

\begin{knitrout}
\definecolor{shadecolor}{rgb}{0.969, 0.969, 0.969}\color{fgcolor}\begin{kframe}
\begin{alltt}
\hlcomment{### Name: geom_rect}
\hlcomment{### Title: 2d rectangles.}
\hlcomment{### Aliases: geom_rect}

\hlcomment{### ** Examples}

df <- \hlfunctioncall{data.frame}(
  x = \hlfunctioncall{sample}(10, 20, replace = TRUE),
  y = \hlfunctioncall{sample}(10, 20, replace = TRUE)
)
\hlfunctioncall{ggplot}(df, \hlfunctioncall{aes}(xmin = x, xmax = x + 1, ymin = y, ymax = y + 2)) +
\hlfunctioncall{geom_rect}()
\end{alltt}
\end{kframe}
\includegraphics[width=\maxwidth]{figure/021-ggplot2-geoms-geom_rect} 
\begin{kframe}\begin{alltt}


\end{alltt}
\end{kframe}
\end{knitrout}



\section{geom\_ribbon}

\begin{knitrout}
\definecolor{shadecolor}{rgb}{0.969, 0.969, 0.969}\color{fgcolor}\begin{kframe}
\begin{alltt}
\hlcomment{### Name: geom_ribbon}
\hlcomment{### Title: Ribbons, y range with continuous x values.}
\hlcomment{### Aliases: geom_ribbon}

\hlcomment{### ** Examples}

\hlcomment{## No test: }
\hlcomment{# Generate data}
huron <- \hlfunctioncall{data.frame}(year = 1875:1972, level = \hlfunctioncall{as.vector}(LakeHuron))
\hlfunctioncall{library}(plyr) \hlcomment{# to access round_any}
huron$decade <- \hlfunctioncall{round_any}(huron$year, 10, floor)

h <- \hlfunctioncall{ggplot}(huron, \hlfunctioncall{aes}(x=year))

h + \hlfunctioncall{geom_ribbon}(\hlfunctioncall{aes}(ymin=0, ymax=level))
\end{alltt}
\end{kframe}
\includegraphics[width=\maxwidth]{figure/021-ggplot2-geoms-geom_ribbon1} 
\begin{kframe}\begin{alltt}
h + \hlfunctioncall{geom_area}(\hlfunctioncall{aes}(y = level))
\end{alltt}
\end{kframe}
\includegraphics[width=\maxwidth]{figure/021-ggplot2-geoms-geom_ribbon2} 
\begin{kframe}\begin{alltt}

\hlcomment{# Add aesthetic mappings}
h + \hlfunctioncall{geom_ribbon}(\hlfunctioncall{aes}(ymin=level-1, ymax=level+1))
\end{alltt}
\end{kframe}
\includegraphics[width=\maxwidth]{figure/021-ggplot2-geoms-geom_ribbon3} 
\begin{kframe}\begin{alltt}
h + \hlfunctioncall{geom_ribbon}(\hlfunctioncall{aes}(ymin=level-1, ymax=level+1)) + \hlfunctioncall{geom_line}(\hlfunctioncall{aes}(y=level))
\end{alltt}
\end{kframe}
\includegraphics[width=\maxwidth]{figure/021-ggplot2-geoms-geom_ribbon4} 
\begin{kframe}\begin{alltt}

\hlcomment{# Take out some values in the middle for an example of NA handling}
huron[huron$year > 1900 & huron$year < 1910, \hlstring{"level"}] <- NA
h <- \hlfunctioncall{ggplot}(huron, \hlfunctioncall{aes}(x=year))
h + \hlfunctioncall{geom_ribbon}(\hlfunctioncall{aes}(ymin=level-1, ymax=level+1)) + \hlfunctioncall{geom_line}(\hlfunctioncall{aes}(y=level))
\end{alltt}
\end{kframe}
\includegraphics[width=\maxwidth]{figure/021-ggplot2-geoms-geom_ribbon5} 
\begin{kframe}\begin{alltt}

\hlcomment{# Another data set, with multiple y's for each x}
m <- \hlfunctioncall{ggplot}(movies, \hlfunctioncall{aes}(y=votes, x=year))
(m <- m + \hlfunctioncall{geom_point}())
\end{alltt}
\end{kframe}
\includegraphics[width=\maxwidth]{figure/021-ggplot2-geoms-geom_ribbon6} 
\begin{kframe}\begin{alltt}

\hlcomment{# The default summary isn't that useful}
m + \hlfunctioncall{stat_summary}(geom=\hlstring{"ribbon"}, fun.ymin=\hlstring{"min"}, fun.ymax=\hlstring{"max"})
\end{alltt}
\end{kframe}
\includegraphics[width=\maxwidth]{figure/021-ggplot2-geoms-geom_ribbon7} 
\begin{kframe}\begin{alltt}
m + \hlfunctioncall{stat_summary}(geom=\hlstring{"ribbon"}, fun.data=\hlstring{"median_hilow"})
\end{alltt}
\end{kframe}
\includegraphics[width=\maxwidth]{figure/021-ggplot2-geoms-geom_ribbon8} 
\begin{kframe}\begin{alltt}

\hlcomment{# Use qplot instead}
\hlfunctioncall{qplot}(year, level, data=huron, geom=\hlfunctioncall{c}(\hlstring{"area"}, \hlstring{"line"}))
\end{alltt}


{\ttfamily\noindent\textcolor{warningcolor}{\#\# Warning: Removed 9 rows containing missing values (position\_stack).}}\end{kframe}
\includegraphics[width=\maxwidth]{figure/021-ggplot2-geoms-geom_ribbon9} 
\begin{kframe}\begin{alltt}
\hlcomment{## End(No test)}


\end{alltt}
\end{kframe}
\end{knitrout}



\section{geom\_rug}

\begin{knitrout}
\definecolor{shadecolor}{rgb}{0.969, 0.969, 0.969}\color{fgcolor}\begin{kframe}
\begin{alltt}
\hlcomment{### Name: geom_rug}
\hlcomment{### Title: Marginal rug plots.}
\hlcomment{### Aliases: geom_rug}

\hlcomment{### ** Examples}

p <- \hlfunctioncall{ggplot}(mtcars, \hlfunctioncall{aes}(x=wt, y=mpg))
p + \hlfunctioncall{geom_point}()
\end{alltt}
\end{kframe}
\includegraphics[width=\maxwidth]{figure/021-ggplot2-geoms-geom_rug1} 
\begin{kframe}\begin{alltt}
p + \hlfunctioncall{geom_point}() + \hlfunctioncall{geom_rug}()
\end{alltt}
\end{kframe}
\includegraphics[width=\maxwidth]{figure/021-ggplot2-geoms-geom_rug2} 
\begin{kframe}\begin{alltt}
p + \hlfunctioncall{geom_point}() + \hlfunctioncall{geom_rug}(sides=\hlstring{"b"})    # Rug on bottom only
\end{alltt}
\end{kframe}
\includegraphics[width=\maxwidth]{figure/021-ggplot2-geoms-geom_rug3} 
\begin{kframe}\begin{alltt}
p + \hlfunctioncall{geom_point}() + \hlfunctioncall{geom_rug}(sides=\hlstring{"trbl"}) # All four sides
\end{alltt}
\end{kframe}
\includegraphics[width=\maxwidth]{figure/021-ggplot2-geoms-geom_rug4} 
\begin{kframe}\begin{alltt}
p + \hlfunctioncall{geom_point}() + \hlfunctioncall{geom_rug}(position=\hlstring{'jitter'})
\end{alltt}
\end{kframe}
\includegraphics[width=\maxwidth]{figure/021-ggplot2-geoms-geom_rug5} 
\begin{kframe}\begin{alltt}


\end{alltt}
\end{kframe}
\end{knitrout}



\section{geom\_segment}

\begin{knitrout}
\definecolor{shadecolor}{rgb}{0.969, 0.969, 0.969}\color{fgcolor}\begin{kframe}
\begin{alltt}
\hlcomment{### Name: geom_segment}
\hlcomment{### Title: Single line segments.}
\hlcomment{### Aliases: geom_segment}

\hlcomment{### ** Examples}

\hlfunctioncall{library}(grid) \hlcomment{# needed for arrow function}
p <- \hlfunctioncall{ggplot}(seals, \hlfunctioncall{aes}(x = long, y = lat))
(p <- p + \hlfunctioncall{geom_segment}(\hlfunctioncall{aes}(xend = long + delta_long, yend = lat + delta_lat), arrow = \hlfunctioncall{arrow}(length = \hlfunctioncall{unit}(0.1,\hlstring{"cm"}))))
\end{alltt}
\end{kframe}
\includegraphics[width=\maxwidth]{figure/021-ggplot2-geoms-geom_segment1} 
\begin{kframe}\begin{alltt}

\hlfunctioncall{if} (\hlfunctioncall{require}(\hlstring{"maps"})) \{

xlim <- \hlfunctioncall{range}(seals$long)
ylim <- \hlfunctioncall{range}(seals$lat)
usamap <- \hlfunctioncall{data.frame}(\hlfunctioncall{map}(\hlstring{"world"}, xlim = xlim, ylim = ylim, plot =
FALSE)[\hlfunctioncall{c}(\hlstring{"x"},\hlstring{"y"})])
usamap <- \hlfunctioncall{rbind}(usamap, NA, \hlfunctioncall{data.frame}(\hlfunctioncall{map}(\hlstring{'state'}, xlim = xlim, ylim
= ylim, plot = FALSE)[\hlfunctioncall{c}(\hlstring{"x"},\hlstring{"y"})]))
\hlfunctioncall{names}(usamap) <- \hlfunctioncall{c}(\hlstring{"long"}, \hlstring{"lat"})

p + \hlfunctioncall{geom_path}(data = usamap) + \hlfunctioncall{scale_x_continuous}(limits = xlim)
\}
\end{alltt}
\end{kframe}
\includegraphics[width=\maxwidth]{figure/021-ggplot2-geoms-geom_segment2} 
\begin{kframe}\begin{alltt}

\hlcomment{# You can also use geom_segment to recreate plot(type = "h") :}
counts <- \hlfunctioncall{as.data.frame}(\hlfunctioncall{table}(x = \hlfunctioncall{rpois}(100,5)))
counts$x <- \hlfunctioncall{as.numeric}(\hlfunctioncall{as.character}(counts$x))
\hlfunctioncall{with}(counts, \hlfunctioncall{plot}(x, Freq, type = \hlstring{"h"}, lwd = 10))
\end{alltt}
\end{kframe}
\includegraphics[width=\maxwidth]{figure/021-ggplot2-geoms-geom_segment3} 
\begin{kframe}\begin{alltt}

\hlfunctioncall{qplot}(x, Freq, data = counts, geom = \hlstring{"segment"},
  yend = 0, xend = x, size = \hlfunctioncall{I}(10))
\end{alltt}
\end{kframe}
\includegraphics[width=\maxwidth]{figure/021-ggplot2-geoms-geom_segment4} 
\begin{kframe}\begin{alltt}

\hlcomment{# Adding line segments}
\hlfunctioncall{library}(grid) \hlcomment{# needed for arrow function}
b <- \hlfunctioncall{ggplot}(mtcars, \hlfunctioncall{aes}(wt, mpg)) + \hlfunctioncall{geom_point}()
b + \hlfunctioncall{geom_segment}(\hlfunctioncall{aes}(x = 2, y = 15, xend = 2, yend = 25))
\end{alltt}
\end{kframe}
\includegraphics[width=\maxwidth]{figure/021-ggplot2-geoms-geom_segment5} 
\begin{kframe}\begin{alltt}
b + \hlfunctioncall{geom_segment}(\hlfunctioncall{aes}(x = 2, y = 15, xend = 3, yend = 15))
\end{alltt}
\end{kframe}
\includegraphics[width=\maxwidth]{figure/021-ggplot2-geoms-geom_segment6} 
\begin{kframe}\begin{alltt}
b + \hlfunctioncall{geom_segment}(\hlfunctioncall{aes}(x = 5, y = 30, xend = 3.5, yend = 25), arrow = \hlfunctioncall{arrow}(length = \hlfunctioncall{unit}(0.5, \hlstring{"cm"})))
\end{alltt}
\end{kframe}
\includegraphics[width=\maxwidth]{figure/021-ggplot2-geoms-geom_segment7} 
\begin{kframe}\begin{alltt}


\end{alltt}
\end{kframe}
\end{knitrout}



\section{geom\_smooth}

\begin{knitrout}
\definecolor{shadecolor}{rgb}{0.969, 0.969, 0.969}\color{fgcolor}\begin{kframe}
\begin{alltt}
\hlcomment{### Name: geom_smooth}
\hlcomment{### Title: Add a smoothed conditional mean.}
\hlcomment{### Aliases: geom_smooth}

\hlcomment{### ** Examples}

\hlcomment{# See stat_smooth for examples of using built in model fitting}
\hlcomment{# if you need some more flexible, this example shows you how to}
\hlcomment{# plot the fits from any model of your choosing}
\hlfunctioncall{qplot}(wt, mpg, data=mtcars, colour=\hlfunctioncall{factor}(cyl))
\end{alltt}
\end{kframe}
\includegraphics[width=\maxwidth]{figure/021-ggplot2-geoms-geom_smooth1} 
\begin{kframe}\begin{alltt}

model <- \hlfunctioncall{lm}(mpg ~ wt + \hlfunctioncall{factor}(cyl), data=mtcars)
grid <- \hlfunctioncall{with}(mtcars, \hlfunctioncall{expand.grid}(
  wt = \hlfunctioncall{seq}(\hlfunctioncall{min}(wt), \hlfunctioncall{max}(wt), length = 20),
  cyl = \hlfunctioncall{levels}(\hlfunctioncall{factor}(cyl))
))

grid$mpg <- stats::\hlfunctioncall{predict}(model, newdata=grid)

\hlfunctioncall{qplot}(wt, mpg, data=mtcars, colour=\hlfunctioncall{factor}(cyl)) + \hlfunctioncall{geom_line}(data=grid)
\end{alltt}
\end{kframe}
\includegraphics[width=\maxwidth]{figure/021-ggplot2-geoms-geom_smooth2} 
\begin{kframe}\begin{alltt}

\hlcomment{# or with standard errors}

err <- stats::\hlfunctioncall{predict}(model, newdata=grid, se = TRUE)
grid$ucl <- err$fit + 1.96 * err$se.fit
grid$lcl <- err$fit - 1.96 * err$se.fit

\hlfunctioncall{qplot}(wt, mpg, data=mtcars, colour=\hlfunctioncall{factor}(cyl)) +
  \hlfunctioncall{geom_smooth}(\hlfunctioncall{aes}(ymin = lcl, ymax = ucl), data=grid, stat=\hlstring{"identity"})
\end{alltt}
\end{kframe}
\includegraphics[width=\maxwidth]{figure/021-ggplot2-geoms-geom_smooth3} 
\begin{kframe}\begin{alltt}


\end{alltt}
\end{kframe}
\end{knitrout}



\section{geom\_step}

\begin{knitrout}
\definecolor{shadecolor}{rgb}{0.969, 0.969, 0.969}\color{fgcolor}\begin{kframe}
\begin{alltt}
\hlcomment{### Name: geom_step}
\hlcomment{### Title: Connect observations by stairs.}
\hlcomment{### Aliases: geom_step}

\hlcomment{### ** Examples}

\hlcomment{# Simple quantiles/ECDF from examples(plot)}
x <- \hlfunctioncall{sort}(\hlfunctioncall{rnorm}(47))
\hlfunctioncall{qplot}(\hlfunctioncall{seq_along}(x), x, geom=\hlstring{"step"})
\end{alltt}
\end{kframe}
\includegraphics[width=\maxwidth]{figure/021-ggplot2-geoms-geom_step1} 
\begin{kframe}\begin{alltt}

\hlcomment{# Steps go horizontally, then vertically (default)}
\hlfunctioncall{qplot}(\hlfunctioncall{seq_along}(x), x, geom=\hlstring{"step"}, direction = \hlstring{"hv"})
\end{alltt}
\end{kframe}
\includegraphics[width=\maxwidth]{figure/021-ggplot2-geoms-geom_step2} 
\begin{kframe}\begin{alltt}
\hlfunctioncall{plot}(x, type = \hlstring{"s"})
\end{alltt}
\end{kframe}
\includegraphics[width=\maxwidth]{figure/021-ggplot2-geoms-geom_step3} 
\begin{kframe}\begin{alltt}
\hlcomment{# Steps go vertically, then horizontally}
\hlfunctioncall{qplot}(\hlfunctioncall{seq_along}(x), x, geom=\hlstring{"step"}, direction = \hlstring{"vh"})
\end{alltt}
\end{kframe}
\includegraphics[width=\maxwidth]{figure/021-ggplot2-geoms-geom_step4} 
\begin{kframe}\begin{alltt}
\hlfunctioncall{plot}(x, type = \hlstring{"S"})
\end{alltt}
\end{kframe}
\includegraphics[width=\maxwidth]{figure/021-ggplot2-geoms-geom_step5} 
\begin{kframe}\begin{alltt}

\hlcomment{# Also works with other aesthetics}
df <- \hlfunctioncall{data.frame}(
  x = \hlfunctioncall{sort}(\hlfunctioncall{rnorm}(50)),
  trt = \hlfunctioncall{sample}(\hlfunctioncall{c}(\hlstring{"a"}, \hlstring{"b"}), 50, rep = TRUE)
)
\hlfunctioncall{qplot}(\hlfunctioncall{seq_along}(x), x, data = df, geom=\hlstring{"step"}, colour = trt)
\end{alltt}
\end{kframe}
\includegraphics[width=\maxwidth]{figure/021-ggplot2-geoms-geom_step6} 
\begin{kframe}\begin{alltt}


\end{alltt}
\end{kframe}
\end{knitrout}



\section{geom\_text}

\begin{knitrout}
\definecolor{shadecolor}{rgb}{0.969, 0.969, 0.969}\color{fgcolor}\begin{kframe}
\begin{alltt}
\hlcomment{### Name: geom_text}
\hlcomment{### Title: Textual annotations.}
\hlcomment{### Aliases: geom_text}

\hlcomment{### ** Examples}

\hlcomment{## No test: }
p <- \hlfunctioncall{ggplot}(mtcars, \hlfunctioncall{aes}(x=wt, y=mpg, label=\hlfunctioncall{rownames}(mtcars)))

p + \hlfunctioncall{geom_text}()
\end{alltt}
\end{kframe}
\includegraphics[width=\maxwidth]{figure/021-ggplot2-geoms-geom_text1} 
\begin{kframe}\begin{alltt}
\hlcomment{# Change size of the label}
p + \hlfunctioncall{geom_text}(size=10)
\end{alltt}
\end{kframe}
\includegraphics[width=\maxwidth]{figure/021-ggplot2-geoms-geom_text2} 
\begin{kframe}\begin{alltt}
p <- p + \hlfunctioncall{geom_point}()

\hlcomment{# Set aesthetics to fixed value}
p + \hlfunctioncall{geom_text}()
\end{alltt}
\end{kframe}
\includegraphics[width=\maxwidth]{figure/021-ggplot2-geoms-geom_text3} 
\begin{kframe}\begin{alltt}
p + \hlfunctioncall{geom_point}() + \hlfunctioncall{geom_text}(hjust=0, vjust=0)
\end{alltt}
\end{kframe}
\includegraphics[width=\maxwidth]{figure/021-ggplot2-geoms-geom_text4} 
\begin{kframe}\begin{alltt}
p + \hlfunctioncall{geom_point}() + \hlfunctioncall{geom_text}(angle = 45)
\end{alltt}
\end{kframe}
\includegraphics[width=\maxwidth]{figure/021-ggplot2-geoms-geom_text5} 
\begin{kframe}\begin{alltt}

\hlcomment{# Add aesthetic mappings}
p + \hlfunctioncall{geom_text}(\hlfunctioncall{aes}(colour=\hlfunctioncall{factor}(cyl)))
\end{alltt}
\end{kframe}
\includegraphics[width=\maxwidth]{figure/021-ggplot2-geoms-geom_text6} 
\begin{kframe}\begin{alltt}
p + \hlfunctioncall{geom_text}(\hlfunctioncall{aes}(colour=\hlfunctioncall{factor}(cyl))) + \hlfunctioncall{scale_colour_discrete}(l=40)
\end{alltt}
\end{kframe}
\includegraphics[width=\maxwidth]{figure/021-ggplot2-geoms-geom_text7} 
\begin{kframe}\begin{alltt}

p + \hlfunctioncall{geom_text}(\hlfunctioncall{aes}(size=wt))
\end{alltt}
\end{kframe}
\includegraphics[width=\maxwidth]{figure/021-ggplot2-geoms-geom_text8} 
\begin{kframe}\begin{alltt}
p + \hlfunctioncall{geom_text}(\hlfunctioncall{aes}(size=wt)) + \hlfunctioncall{scale_size}(range=\hlfunctioncall{c}(3,6))
\end{alltt}
\end{kframe}
\includegraphics[width=\maxwidth]{figure/021-ggplot2-geoms-geom_text9} 
\begin{kframe}\begin{alltt}

\hlcomment{# You can display expressions by setting parse = TRUE.  The}
\hlcomment{# details of the display are described in ?plotmath, but note that}
\hlcomment{# geom_text uses strings, not expressions.}
p + \hlfunctioncall{geom_text}(\hlfunctioncall{aes}(label = \hlfunctioncall{paste}(wt, \hlstring{"^("}, cyl, \hlstring{")"}, sep = \hlstring{""})),
  parse = TRUE)
\end{alltt}
\end{kframe}
\includegraphics[width=\maxwidth]{figure/021-ggplot2-geoms-geom_text10} 
\begin{kframe}\begin{alltt}

\hlcomment{# Add an annotation not from a variable source}
c <- \hlfunctioncall{ggplot}(mtcars, \hlfunctioncall{aes}(wt, mpg)) + \hlfunctioncall{geom_point}()
c + \hlfunctioncall{geom_text}(data = NULL, x = 5, y = 30, label = \hlstring{"plot mpg vs. wt"})
\end{alltt}
\end{kframe}
\includegraphics[width=\maxwidth]{figure/021-ggplot2-geoms-geom_text11} 
\begin{kframe}\begin{alltt}
\hlcomment{# Or, you can use annotate}
c + \hlfunctioncall{annotate}(\hlstring{"text"}, label = \hlstring{"plot mpg vs. wt"}, x = 2, y = 15, size = 8, colour = \hlstring{"red"})
\end{alltt}
\end{kframe}
\includegraphics[width=\maxwidth]{figure/021-ggplot2-geoms-geom_text12} 
\begin{kframe}\begin{alltt}

\hlcomment{# Use qplot instead}
\hlfunctioncall{qplot}(wt, mpg, data = mtcars, label = \hlfunctioncall{rownames}(mtcars),
   geom=\hlfunctioncall{c}(\hlstring{"point"}, \hlstring{"text"}))
\end{alltt}
\end{kframe}
\includegraphics[width=\maxwidth]{figure/021-ggplot2-geoms-geom_text13} 
\begin{kframe}\begin{alltt}
\hlfunctioncall{qplot}(wt, mpg, data = mtcars, label = \hlfunctioncall{rownames}(mtcars), size = wt) +
  \hlfunctioncall{geom_text}(colour = \hlstring{"red"})
\end{alltt}
\end{kframe}
\includegraphics[width=\maxwidth]{figure/021-ggplot2-geoms-geom_text14} 
\begin{kframe}\begin{alltt}

\hlcomment{# You can specify family, fontface and lineheight}
p <- \hlfunctioncall{ggplot}(mtcars, \hlfunctioncall{aes}(x=wt, y=mpg, label=\hlfunctioncall{rownames}(mtcars)))
p + \hlfunctioncall{geom_text}(fontface=3)
\end{alltt}
\end{kframe}
\includegraphics[width=\maxwidth]{figure/021-ggplot2-geoms-geom_text15} 
\begin{kframe}\begin{alltt}
p + \hlfunctioncall{geom_text}(\hlfunctioncall{aes}(fontface=am+1))
\end{alltt}
\end{kframe}
\includegraphics[width=\maxwidth]{figure/021-ggplot2-geoms-geom_text16} 
\begin{kframe}\begin{alltt}
p + \hlfunctioncall{geom_text}(\hlfunctioncall{aes}(family=\hlfunctioncall{c}(\hlstring{"serif"}, \hlstring{"mono"})[am+1]))
\end{alltt}
\end{kframe}
\includegraphics[width=\maxwidth]{figure/021-ggplot2-geoms-geom_text17} 
\begin{kframe}\begin{alltt}
\hlcomment{## End(No test)}


\end{alltt}
\end{kframe}
\end{knitrout}



\section{geom\_tile}

\begin{knitrout}
\definecolor{shadecolor}{rgb}{0.969, 0.969, 0.969}\color{fgcolor}\begin{kframe}
\begin{alltt}
\hlcomment{### Name: geom_tile}
\hlcomment{### Title: Tile plane with rectangles.}
\hlcomment{### Aliases: geom_tile}

\hlcomment{### ** Examples}

\hlcomment{## No test: }
\hlcomment{# Generate data}
pp <- \hlfunctioncall{function} (n,r=4) \{
 x <- \hlfunctioncall{seq}(-r*pi, r*pi, len=n)
 df <- \hlfunctioncall{expand.grid}(x=x, y=x)
 df$r <- \hlfunctioncall{sqrt}(df$x^2 + df$y^2)
 df$z <- \hlfunctioncall{cos}(df$r^2)*\hlfunctioncall{exp}(-df$r/6)
 df
\}
p <- \hlfunctioncall{ggplot}(\hlfunctioncall{pp}(20), \hlfunctioncall{aes}(x=x,y=y))

p + \hlfunctioncall{geom_tile}() \hlcomment{#pretty useless!}
\end{alltt}
\end{kframe}
\includegraphics[width=\maxwidth]{figure/021-ggplot2-geoms-geom_tile1} 
\begin{kframe}\begin{alltt}

\hlcomment{# Add aesthetic mappings}
p + \hlfunctioncall{geom_tile}(\hlfunctioncall{aes}(fill=z))
\end{alltt}
\end{kframe}
\includegraphics[width=\maxwidth]{figure/021-ggplot2-geoms-geom_tile2} 
\begin{kframe}\begin{alltt}

\hlcomment{# Change scale}
p + \hlfunctioncall{geom_tile}(\hlfunctioncall{aes}(fill=z)) + \hlfunctioncall{scale_fill_gradient}(low=\hlstring{"green"}, high=\hlstring{"red"})
\end{alltt}
\end{kframe}
\includegraphics[width=\maxwidth]{figure/021-ggplot2-geoms-geom_tile3} 
\begin{kframe}\begin{alltt}

\hlcomment{# Use qplot instead}
\hlfunctioncall{qplot}(x, y, data=\hlfunctioncall{pp}(20), geom=\hlstring{"tile"}, fill=z)
\end{alltt}
\end{kframe}
\includegraphics[width=\maxwidth]{figure/021-ggplot2-geoms-geom_tile4} 
\begin{kframe}\begin{alltt}
\hlfunctioncall{qplot}(x, y, data=\hlfunctioncall{pp}(100), geom=\hlstring{"tile"}, fill=z)
\end{alltt}
\end{kframe}
\includegraphics[width=\maxwidth]{figure/021-ggplot2-geoms-geom_tile5} 
\begin{kframe}\begin{alltt}

\hlcomment{# Missing values}
p <- \hlfunctioncall{ggplot}(\hlfunctioncall{pp}(20)[\hlfunctioncall{sample}(20*20, size=200),], \hlfunctioncall{aes}(x=x,y=y,fill=z))
p + \hlfunctioncall{geom_tile}()
\end{alltt}
\end{kframe}
\includegraphics[width=\maxwidth]{figure/021-ggplot2-geoms-geom_tile6} 
\begin{kframe}\begin{alltt}

\hlcomment{# Input that works with image}
\hlfunctioncall{image}(\hlfunctioncall{t}(volcano)[\hlfunctioncall{ncol}(volcano):1,])
\end{alltt}
\end{kframe}
\includegraphics[width=\maxwidth]{figure/021-ggplot2-geoms-geom_tile7} 
\begin{kframe}\begin{alltt}
\hlfunctioncall{library}(reshape2) \hlcomment{# for melt}
\hlfunctioncall{ggplot}(\hlfunctioncall{melt}(volcano), \hlfunctioncall{aes}(x=Var1, y=Var2, fill=value)) + \hlfunctioncall{geom_tile}()
\end{alltt}
\end{kframe}
\includegraphics[width=\maxwidth]{figure/021-ggplot2-geoms-geom_tile8} 
\begin{kframe}\begin{alltt}

\hlcomment{# inspired by the image-density plots of Ken Knoblauch}
cars <- \hlfunctioncall{ggplot}(mtcars, \hlfunctioncall{aes}(y=\hlfunctioncall{factor}(cyl), x=mpg))
cars + \hlfunctioncall{geom_point}()
\end{alltt}
\end{kframe}
\includegraphics[width=\maxwidth]{figure/021-ggplot2-geoms-geom_tile9} 
\begin{kframe}\begin{alltt}
cars + \hlfunctioncall{stat_bin}(\hlfunctioncall{aes}(fill=..count..), geom=\hlstring{"tile"}, binwidth=3, position=\hlstring{"identity"})
\end{alltt}
\end{kframe}
\includegraphics[width=\maxwidth]{figure/021-ggplot2-geoms-geom_tile10} 
\begin{kframe}\begin{alltt}
cars + \hlfunctioncall{stat_bin}(\hlfunctioncall{aes}(fill=..density..), geom=\hlstring{"tile"}, binwidth=3, position=\hlstring{"identity"})
\end{alltt}
\end{kframe}
\includegraphics[width=\maxwidth]{figure/021-ggplot2-geoms-geom_tile11} 
\begin{kframe}\begin{alltt}

cars + \hlfunctioncall{stat_density}(\hlfunctioncall{aes}(fill=..density..), geom=\hlstring{"tile"}, position=\hlstring{"identity"})
\end{alltt}
\end{kframe}
\includegraphics[width=\maxwidth]{figure/021-ggplot2-geoms-geom_tile12} 
\begin{kframe}\begin{alltt}
cars + \hlfunctioncall{stat_density}(\hlfunctioncall{aes}(fill=..count..), geom=\hlstring{"tile"}, position=\hlstring{"identity"})
\end{alltt}
\end{kframe}
\includegraphics[width=\maxwidth]{figure/021-ggplot2-geoms-geom_tile13} 
\begin{kframe}\begin{alltt}

\hlcomment{# Another example with with unequal tile sizes}
x.cell.boundary <- \hlfunctioncall{c}(0, 4, 6, 8, 10, 14)
example <- \hlfunctioncall{data.frame}(
  x = \hlfunctioncall{rep}(\hlfunctioncall{c}(2, 5, 7, 9, 12), 2),
  y = \hlfunctioncall{factor}(\hlfunctioncall{rep}(\hlfunctioncall{c}(1,2), each=5)),
  z = \hlfunctioncall{rep}(1:5, each=2),
  w = \hlfunctioncall{rep}(\hlfunctioncall{diff}(x.cell.boundary), 2)
)

\hlfunctioncall{qplot}(x, y, fill=z, data=example, geom=\hlstring{"tile"})
\end{alltt}
\end{kframe}
\includegraphics[width=\maxwidth]{figure/021-ggplot2-geoms-geom_tile14} 
\begin{kframe}\begin{alltt}
\hlfunctioncall{qplot}(x, y, fill=z, data=example, geom=\hlstring{"tile"}, width=w)
\end{alltt}
\end{kframe}
\includegraphics[width=\maxwidth]{figure/021-ggplot2-geoms-geom_tile15} 
\begin{kframe}\begin{alltt}
\hlfunctioncall{qplot}(x, y, fill=\hlfunctioncall{factor}(z), data=example, geom=\hlstring{"tile"}, width=w)
\end{alltt}
\end{kframe}
\includegraphics[width=\maxwidth]{figure/021-ggplot2-geoms-geom_tile16} 
\begin{kframe}\begin{alltt}

\hlcomment{# You can manually set the colour of the tiles using}
\hlcomment{# scale_manual}
col <- \hlfunctioncall{c}(\hlstring{"darkblue"}, \hlstring{"blue"}, \hlstring{"green"}, \hlstring{"orange"}, \hlstring{"red"})
\hlfunctioncall{qplot}(x, y, fill=col[z], data=example, geom=\hlstring{"tile"}, width=w, group=1) + \hlfunctioncall{scale_fill_identity}(labels=letters[1:5], breaks=col)
\end{alltt}
\end{kframe}
\includegraphics[width=\maxwidth]{figure/021-ggplot2-geoms-geom_tile17} 
\begin{kframe}\begin{alltt}
\hlcomment{## End(No test)}


\end{alltt}
\end{kframe}
\end{knitrout}



\section{geom\_violin}

\begin{knitrout}
\definecolor{shadecolor}{rgb}{0.969, 0.969, 0.969}\color{fgcolor}\begin{kframe}
\begin{alltt}
\hlcomment{### Name: geom_violin}
\hlcomment{### Title: Violin plot.}
\hlcomment{### Aliases: geom_violin}

\hlcomment{### ** Examples}

\hlcomment{## No test: }
p <- \hlfunctioncall{ggplot}(mtcars, \hlfunctioncall{aes}(\hlfunctioncall{factor}(cyl), mpg))

p + \hlfunctioncall{geom_violin}()
\end{alltt}
\end{kframe}
\includegraphics[width=\maxwidth]{figure/021-ggplot2-geoms-geom_violin1} 
\begin{kframe}\begin{alltt}
\hlfunctioncall{qplot}(\hlfunctioncall{factor}(cyl), mpg, data = mtcars, geom = \hlstring{"violin"})
\end{alltt}
\end{kframe}
\includegraphics[width=\maxwidth]{figure/021-ggplot2-geoms-geom_violin2} 
\begin{kframe}\begin{alltt}

p + \hlfunctioncall{geom_violin}() + \hlfunctioncall{geom_jitter}(height = 0)
\end{alltt}
\end{kframe}
\includegraphics[width=\maxwidth]{figure/021-ggplot2-geoms-geom_violin3} 
\begin{kframe}\begin{alltt}
p + \hlfunctioncall{geom_violin}() + \hlfunctioncall{coord_flip}()
\end{alltt}
\end{kframe}
\includegraphics[width=\maxwidth]{figure/021-ggplot2-geoms-geom_violin4} 
\begin{kframe}\begin{alltt}
\hlfunctioncall{qplot}(\hlfunctioncall{factor}(cyl), mpg, data = mtcars, geom = \hlstring{"violin"}) +
  \hlfunctioncall{coord_flip}()
\end{alltt}
\end{kframe}
\includegraphics[width=\maxwidth]{figure/021-ggplot2-geoms-geom_violin5} 
\begin{kframe}\begin{alltt}

\hlcomment{# Scale maximum width proportional to sample size:}
p + \hlfunctioncall{geom_violin}(scale = \hlstring{"count"})
\end{alltt}
\end{kframe}
\includegraphics[width=\maxwidth]{figure/021-ggplot2-geoms-geom_violin6} 
\begin{kframe}\begin{alltt}

\hlcomment{# Default is to trim violins to the range of the data. To disable:}
p + \hlfunctioncall{geom_violin}(trim = FALSE)
\end{alltt}
\end{kframe}
\includegraphics[width=\maxwidth]{figure/021-ggplot2-geoms-geom_violin7} 
\begin{kframe}\begin{alltt}

\hlcomment{# Use a smaller bandwidth for closer density fit (default is 1).}
p + \hlfunctioncall{geom_violin}(adjust = .5)
\end{alltt}
\end{kframe}
\includegraphics[width=\maxwidth]{figure/021-ggplot2-geoms-geom_violin8} 
\begin{kframe}\begin{alltt}

\hlcomment{# Add aesthetic mappings}
\hlcomment{# Note that violins are automatically dodged when any aesthetic is}
\hlcomment{# a factor}
p + \hlfunctioncall{geom_violin}(\hlfunctioncall{aes}(fill = cyl))
\end{alltt}
\end{kframe}
\includegraphics[width=\maxwidth]{figure/021-ggplot2-geoms-geom_violin9} 
\begin{kframe}\begin{alltt}
p + \hlfunctioncall{geom_violin}(\hlfunctioncall{aes}(fill = \hlfunctioncall{factor}(cyl)))
\end{alltt}
\end{kframe}
\includegraphics[width=\maxwidth]{figure/021-ggplot2-geoms-geom_violin10} 
\begin{kframe}\begin{alltt}
p + \hlfunctioncall{geom_violin}(\hlfunctioncall{aes}(fill = \hlfunctioncall{factor}(vs)))
\end{alltt}
\end{kframe}
\includegraphics[width=\maxwidth]{figure/021-ggplot2-geoms-geom_violin11} 
\begin{kframe}\begin{alltt}
p + \hlfunctioncall{geom_violin}(\hlfunctioncall{aes}(fill = \hlfunctioncall{factor}(am)))
\end{alltt}
\end{kframe}
\includegraphics[width=\maxwidth]{figure/021-ggplot2-geoms-geom_violin12} 
\begin{kframe}\begin{alltt}

\hlcomment{# Set aesthetics to fixed value}
p + \hlfunctioncall{geom_violin}(fill = \hlstring{"grey80"}, colour = \hlstring{"#3366FF"})
\end{alltt}
\end{kframe}
\includegraphics[width=\maxwidth]{figure/021-ggplot2-geoms-geom_violin13} 
\begin{kframe}\begin{alltt}
\hlfunctioncall{qplot}(\hlfunctioncall{factor}(cyl), mpg, data = mtcars, geom = \hlstring{"violin"},
  colour = \hlfunctioncall{I}(\hlstring{"#3366FF"}))
\end{alltt}
\end{kframe}
\includegraphics[width=\maxwidth]{figure/021-ggplot2-geoms-geom_violin14} 
\begin{kframe}\begin{alltt}

\hlcomment{# Scales vs. coordinate transforms -------}
\hlcomment{# Scale transformations occur before the density statistics are computed.}
\hlcomment{# Coordinate transformations occur afterwards.  Observe the effect on the}
\hlcomment{# number of outliers.}
\hlfunctioncall{library}(plyr) \hlcomment{# to access round_any}
m <- \hlfunctioncall{ggplot}(movies, \hlfunctioncall{aes}(y = votes, x = rating,
   group = \hlfunctioncall{round_any}(rating, 0.5)))
m + \hlfunctioncall{geom_violin}()
\end{alltt}


{\ttfamily\noindent\textcolor{warningcolor}{\#\# Warning: position\_dodge requires constant width: output may be incorrect}}\end{kframe}
\includegraphics[width=\maxwidth]{figure/021-ggplot2-geoms-geom_violin15} 
\begin{kframe}\begin{alltt}
m + \hlfunctioncall{geom_violin}() + \hlfunctioncall{scale_y_log10}()
\end{alltt}


{\ttfamily\noindent\textcolor{warningcolor}{\#\# Warning: position\_dodge requires constant width: output may be incorrect}}\end{kframe}
\includegraphics[width=\maxwidth]{figure/021-ggplot2-geoms-geom_violin16} 
\begin{kframe}\begin{alltt}
m + \hlfunctioncall{geom_violin}() + \hlfunctioncall{coord_trans}(y = \hlstring{"log10"})
\end{alltt}


{\ttfamily\noindent\textcolor{warningcolor}{\#\# Warning: position\_dodge requires constant width: output may be incorrect}}\end{kframe}
\includegraphics[width=\maxwidth]{figure/021-ggplot2-geoms-geom_violin17} 
\begin{kframe}\begin{alltt}
m + \hlfunctioncall{geom_violin}() + \hlfunctioncall{scale_y_log10}() + \hlfunctioncall{coord_trans}(y = \hlstring{"log10"})
\end{alltt}


{\ttfamily\noindent\textcolor{warningcolor}{\#\# Warning: position\_dodge requires constant width: output may be incorrect}}\end{kframe}
\includegraphics[width=\maxwidth]{figure/021-ggplot2-geoms-geom_violin18} 
\begin{kframe}\begin{alltt}

\hlcomment{# Violin plots with continuous x:}
\hlcomment{# Use the group aesthetic to group observations in violins}
\hlfunctioncall{qplot}(year, budget, data = movies, geom = \hlstring{"violin"})
\end{alltt}


{\ttfamily\noindent\textcolor{warningcolor}{\#\# Warning: Removed 904 rows containing non-finite values (stat\_ydensity).}}\end{kframe}
\includegraphics[width=\maxwidth]{figure/021-ggplot2-geoms-geom_violin19} 
\begin{kframe}\begin{alltt}
\hlfunctioncall{qplot}(year, budget, data = movies, geom = \hlstring{"violin"},
  group = \hlfunctioncall{round_any}(year, 10, floor))
\end{alltt}


{\ttfamily\noindent\textcolor{warningcolor}{\#\# Warning: Removed 904 rows containing non-finite values (stat\_ydensity).}}

{\ttfamily\noindent\textcolor{warningcolor}{\#\# Warning: position\_dodge requires constant width: output may be incorrect}}\end{kframe}
\includegraphics[width=\maxwidth]{figure/021-ggplot2-geoms-geom_violin20} 
\begin{kframe}\begin{alltt}
\hlcomment{## End(No test)}


\end{alltt}
\end{kframe}
\end{knitrout}



\section{geom\_vline}

\begin{knitrout}
\definecolor{shadecolor}{rgb}{0.969, 0.969, 0.969}\color{fgcolor}\begin{kframe}
\begin{alltt}
\hlcomment{### Name: geom_vline}
\hlcomment{### Title: Line, vertical.}
\hlcomment{### Aliases: geom_vline}

\hlcomment{### ** Examples}

\hlcomment{# Fixed lines}
p <- \hlfunctioncall{ggplot}(mtcars, \hlfunctioncall{aes}(x = wt, y = mpg)) + \hlfunctioncall{geom_point}()
p + \hlfunctioncall{geom_vline}(xintercept = 5)
\end{alltt}
\end{kframe}
\includegraphics[width=\maxwidth]{figure/021-ggplot2-geoms-geom_vline1} 
\begin{kframe}\begin{alltt}
p + \hlfunctioncall{geom_vline}(xintercept = 1:5)
\end{alltt}
\end{kframe}
\includegraphics[width=\maxwidth]{figure/021-ggplot2-geoms-geom_vline2} 
\begin{kframe}\begin{alltt}
p + \hlfunctioncall{geom_vline}(xintercept = 1:5, colour=\hlstring{"green"}, linetype = \hlstring{"longdash"})
\end{alltt}
\end{kframe}
\includegraphics[width=\maxwidth]{figure/021-ggplot2-geoms-geom_vline3} 
\begin{kframe}\begin{alltt}
p + \hlfunctioncall{geom_vline}(\hlfunctioncall{aes}(xintercept = wt))
\end{alltt}
\end{kframe}
\includegraphics[width=\maxwidth]{figure/021-ggplot2-geoms-geom_vline4} 
\begin{kframe}\begin{alltt}

\hlcomment{# With coordinate transforms}
p + \hlfunctioncall{geom_vline}(\hlfunctioncall{aes}(xintercept = wt)) + \hlfunctioncall{coord_equal}()
\end{alltt}
\end{kframe}
\includegraphics[width=\maxwidth]{figure/021-ggplot2-geoms-geom_vline5} 
\begin{kframe}\begin{alltt}
p + \hlfunctioncall{geom_vline}(\hlfunctioncall{aes}(xintercept = wt)) + \hlfunctioncall{coord_flip}()
\end{alltt}
\end{kframe}
\includegraphics[width=\maxwidth]{figure/021-ggplot2-geoms-geom_vline6} 
\begin{kframe}\begin{alltt}
p + \hlfunctioncall{geom_vline}(\hlfunctioncall{aes}(xintercept = wt)) + \hlfunctioncall{coord_polar}()
\end{alltt}
\end{kframe}
\includegraphics[width=\maxwidth]{figure/021-ggplot2-geoms-geom_vline7} 
\begin{kframe}\begin{alltt}

p2 <- p + \hlfunctioncall{aes}(colour = \hlfunctioncall{factor}(cyl))
p2 + \hlfunctioncall{geom_vline}(xintercept = 15)
\end{alltt}
\end{kframe}
\includegraphics[width=\maxwidth]{figure/021-ggplot2-geoms-geom_vline8} 
\begin{kframe}\begin{alltt}

\hlcomment{# To display different lines in different facets, you need to}
\hlcomment{# create a data frame.}
p <- \hlfunctioncall{qplot}(mpg, wt, data=mtcars, facets = vs ~ am)
vline.data <- \hlfunctioncall{data.frame}(z = \hlfunctioncall{c}(15, 20, 25, 30), vs = \hlfunctioncall{c}(0, 0, 1, 1), am = \hlfunctioncall{c}(0, 1, 0, 1))
p + \hlfunctioncall{geom_vline}(\hlfunctioncall{aes}(xintercept = z), vline.data)
\end{alltt}
\end{kframe}
\includegraphics[width=\maxwidth]{figure/021-ggplot2-geoms-geom_vline9} 
\begin{kframe}\begin{alltt}


\end{alltt}
\end{kframe}
\end{knitrout}




\end{document}
